%!TEX root = ../../thesis.tex

Correctly modelling signal and background processes is of utmost importance when designing 
a search strategy; this allows differences to be exploited in order to optimise the 
sensitivity. Whereas background modelling can often be experimentally verified (and in 
some cases estimated by entirely data-driven methods), the signal processes must rely 
purely upon theoretical models. When comparing a measured signal to that expected, the 
normalisation of the prediction becomes important.

\begin{figure}[b]
	\includegraphics[width=\mediumfigwidth]{axodraw/ggf_WWlvlv.pdf}
	\caption{Leading order Feynman diagram for gluon-gluon fusion.}
	\label{fig:sig:ggF}
\end{figure}

The analysis described in \Chapter~\ref{chap:selection} is designed to search for 
\ggHWWlvlv, \ie the \ac{ggF} production mode (see \Figure~\ref{fig:sig:ggF}). This 
proceeds via a heavy quark loop, dominated by the top quark, with two factors of \alphaS 
at \ac{LO}. The perturbative series of \ac{ggF} converges poorly; the $K$-factor, defined 
as the ratio to the \ac{LO} cross section $K = \sigma / \sigma_{\text{LO}}$, is typically 
$K \approx 1.7\text{--}1.9$ at \ac{NLO} and $K \approx 2.0\text{--}2.2$ at \ac{NNLO}.
This is believed to be mostly due to the nature of the gluon vertex, rather than 
contributions from new diagrams accessible at higher orders \cite{Becher:2009}. As 
explained in \Section~\ref{sec:qcd:pqcd}, such sizeable higher order corrections lead 
to large theoretical uncertainties. Uncertainties due to \acp{PDF} are also significant, 
since the low-$x$ gluon (which is instrumental in \ac{ggF}) is relatively poorly 
constrained (see \Figure~\ref{fig:qcd:pdf}).

