%!TEX root = ../../thesis.tex

Signal and background modelling is of utmost importance when designing a search strategy; 
differences can be exploited to optimise the sensitivity. Whereas background modelling can 
often be experimentally verified (and in some cases estimated by entirely data-driven 
methods), the signal processes must rely purely upon theoretical models.

\begin{figure}[b]
	\includegraphics[width=\mediumfigwidth]{axodraw/ggf_WWlvlv.pdf}
	\caption{Leading order Feynman diagram for \acf{ggF}.}
	\label{fig:sig:ggF}
\end{figure}

The analysis described in \Chapter~\ref{chap:selection} is designed to search for the 
\acsu{ggF} production mode (see \Figure~\ref{fig:sig:ggF}), which exhibits large 
theoretical uncertainties. \Section~\ref{sec:ggf_inc} briefly outlines the current 
knowledge of the inclusive \ac{ggF} cross section. The significant theoretical issues 
introduced by jet vetoes are considered in \Section~\ref{sec:ggf_jetbin}. Since the \HWW 
analysis is highly exclusive, \meps{\powheg}{\pythia{8}} is used to model the \ac{ggF} 
signal. Uncertainties in the \ptOf{\PHiggs} distribution and the acceptance of the event 
selection are described in \Section~\ref{sec:ggf_pt} and \Section~\ref{sec:ggf_acc} 
respectively.
