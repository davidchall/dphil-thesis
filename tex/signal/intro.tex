%!TEX root = ../../thesis.tex

This thesis describes a search for the ggF production mode (see \Figure~\ref{fig:sig:ggF}). 
This exhibits large theoretical uncertainties due to higher order corrections,\footnote{
	The poor convergence of the ggF perturbative series, with respect to similar 
	\HepProcess{\Pquark\APquark}-initiated processes, is thought to be due to the larger 
	colour factor of the gluon.
}
and so its cross section is calculated at NNLO+NNLL in QCD and NLO in EW (see 
\Figure~\ref{fig:higgs_xs}). PDF uncertainties are also significant, since the low-$x$ gluon 
is relatively poorly constrained (see \Figure~\ref{fig:qcd:pdf}). Calculations are also 
sensitive to the treatment of quark masses in the loop.

\Section~\ref{sec:ggf_jetbin} considers the significant theoretical issues introduced by the 
jet binning of the analysis, and then the MC modelling of ggF is discussed in 
\Section~\ref{sec:ggf_mc}.

\begin{figure}[b]
	\includegraphics[width=\mediumfigwidth]{axodraw/ggf_WWlvlv.pdf}
	\caption{Leading order Feynman diagram for gluon-gluon fusion (ggF).}
	\label{fig:sig:ggF}
\end{figure}
