%!TEX root = ../../thesis.tex

The \ggHWW analysis is binned according to jet multiplicity, in order to exploit the vastly 
different background compositions in each jet bin (see \Figure~\ref{fig:sel:njets}); the 
0-jet, 1-jet and \twojet bins each have dedicated event selection criteria. Uncertainties 
in the expected ggF cross section must be evaluated separately for each jet bin, and 
correlations considered when the bins are combined. Perturbative uncertainties in the jet 
binning itself are considered independently from the other selection criteria, since they 
posses some additional subtleties.



\subsection{Perturbative uncertainties in jet-binned cross sections}
\label{sec:ggF:naive}

Consider splitting a cross section into two parts: an exclusive 0-jet cross section, 
$\sigma_0$, and an inclusive $\geq\!1$-jet cross section, $\sigma_{\geq1}$:
\begin{equation}
	\sigma_{\total} = \sigma_0 \parenths{\ptcut} + \sigma_{\geq1} \parenths{\ptcut}
\end{equation}
where \ptcut is the jet \pt threshold \cite{YR2}. In $\sigma_{\geq1}$, the requirement of 
a jet with $\pt > \ptcut$ introduces double logarithmic contributions 
$\alpha_{\text{S}}^{k+m} L^{2m}$, where $L \sim \ln\parenths{\ptcut/Q}$ and $Q$ is the 
scale of the hard scatter ($Q = \mH$ for ggF). These terms are analogous to the logarithms 
introduced by soft gluon emission (see \Section~\ref{sec:qcd:resum}), though they depend 
upon the process and also the jet algorithm and parameters (\eg \antikt with $R=0.4$).

The schematic structures of the two inclusive cross sections are
\begin{equation}
	\sigma_{\total} &\sim \alpha_{\text{S}}^k \{1 &+& \alphaS &+& \alpha_{\text{S}}^2 &+& \ofOrder{\alpha_{\text{S}}^3}\} \\
	\sigma_{\geq1}  &\sim \alpha_{\text{S}}^k \{&\phantom{{}=+}&\alphaS (L^2 + L + 1) &+& \alpha_{\text{S}}^2 (L^4 + L^3 + L^2 + L + 1) &+& \ofOrder{\alpha_{\text{S}}^3 L^6}\} \,.
\end{equation}
When $\ptcut \ll \mH$, the logarithms can overcome the \alphaS suppression and provide 
significant corrections to $\sigma_{\geq1}$. When similar in size to the perturbative 
corrections to $\sigma_{\total}$, the scale dependence of $\sigma_0 = \sigma_{\total} - 
\sigma_{\geq1}$ is reduced by cancellations between the two series. This suggests that 
na\"{i}vely varying \mur and \muf might underestimate perturbative uncertainties. This is 
confirmed by \Figure~\ref{fig:ggF:naive}, which shows that the cancellations at 
\unit{$\ptcut = 25$}{\GeV} (used in the \HWW analysis) are rather extreme.

\begin{figure}[t]
	\includegraphics[width=\mediumfigwidth]{tex/signal/sigma0_naive}
	\caption{The exclusive 0-jet ggF cross section versus the jet \pt threshold \cite{YR2}. 
	The bands show the perturbative uncertainties evaluated via na\"{i}ve scale variations.}
	\label{fig:ggF:naive}
\end{figure}

When discussing uncertainties in jet-binned cross sections, it is convenient to consider 
a general parametrisation of the covariance matrix. In the \braces{\sigma_0, \sigma_{\geq1}}
basis, the covariance matrix is decomposed into two uncertainty sources
\begin{equation}
	C &= C^{\text{yield}} + C^{\text{migration}} \\
	&= \twomatrix{\parenths{\Delta^{\text{y}}_0}^2 & \Delta^{\text{y}}_0 \Delta^{\text{y}}_{\geq1}}{\Delta^{\text{y}}_0 \Delta^{\text{y}}_{\geq1} & \parenths{\Delta^{\text{y}}_{\geq1}}^2} + \parenths{\Delta_{0\rightarrow}^{\text{mig}}}^2\twomatrix{1 & -1}{-1 & 1} \,.
\end{equation}
The yield component is fully correlated between jet bins, though can affect them with 
different magnitudes. The migration component is fully anti-correlated and affects both 
bins equally, conserving the normalisation. In the limit setting procedure, these sources 
are treated as nuisance parameters with \braces{\sigma_0, \sigma_{\geq1}} uncertainty 
amplitudes
\begin{equation}
	\begin{array}{l@{}l@{}l}
		\begin{array}{r}
			\kappa^{\text{yield}}              : \Big( \\
			\kappa^{\text{mig}}_{0\rightarrow} : \Big(
		\end{array}
		&
		\begin{array}{@{}rrr@{}}
			\Delta^{\text{y}}_0, & \Delta^{\text{y}}_{\geq1} \\
			\Delta^{\text{mig}}_{0\rightarrow}, & -\Delta^{\text{mig}}_{0\rightarrow}
		\end{array}
		&
		\begin{array}{@{}l}
			\Big) \\ \Big) \,.
		\end{array}
	\end{array}
	\label{eq:ggF:2bin_np}
\end{equation}

Different prescriptions for evaluating perturbative uncertainties are defined by their 
choice of $\Delta^{\text{y}}_0$, $\Delta^{\text{y}}_{\geq1}$ and 
$\Delta^{\text{mig}}_{0\rightarrow}$. This includes a choice of observables to measure 
uncertainties in, and also the method of measuring the uncertainties. For example, the 
na\"{i}ve prescription described above is equivalent to choosing
\begin{equation}
	\text{Na\"{i}ve:} 
	\quad\quad \Delta^{\text{y}}_0 &= \Delta\sigma_{0},& 
	\quad\quad \Delta^{\text{y}}_{\geq1} &= \Delta\sigma_{\geq1},&
	\quad\quad \Delta^{\text{mig}}_{0\rightarrow} &= 0
\end{equation}
where uncertainties are evaluated at fixed order by varying \mur and \muf.

In the \HWW analysis, there is also an exclusive 1-jet bin. The second jet veto introduces 
an additional source of migrations, now between the 1-jet and \twojet bins. 
Therefore, in the \braces{\sigma_0, \sigma_1, \sigma_{\geq2}} basis, the covariance 
matrix has three components
\begin{equation}
	C &= 
	\threematrix{
		\parenths{\Delta^{\text{y}}_0}^2 & 
		\Delta^{\text{y}}_0 \Delta^{\text{y}}_1 & 
		\Delta^{\text{y}}_0 \Delta^{\text{y}}_{\geq2}
	}{
		\Delta^{\text{y}}_0 \Delta^{\text{y}}_1 & 
		\parenths{\Delta^{\text{y}}_1}^2 & 
		\Delta^{\text{y}}_1 \Delta^{\text{y}}_{\geq2}
	}{
		\Delta^{\text{y}}_0 \Delta^{\text{y}}_{\geq2} & 
		\Delta^{\text{y}}_1 \Delta^{\text{y}}_{\geq2} & 
		\parenths{\Delta^{\text{y}}_{\geq2}}^2
	}
	\nonumber \\
	&+ \parenths{\Delta^{\text{mig}}_{0\rightarrow}}^2
	\threematrix{
		1 & -\parenths{1-\rho} & -\rho
	}{
		-\parenths{1-\rho} & \parenths{1-\rho}^2 & \rho\parenths{1-\rho}
	}{
		-\rho & \rho\parenths{1-\rho} & \rho^2
	}
	+ \parenths{\Delta^{\text{mig}}_{1\rightarrow}}^2
	\threematrix{
		0 & 0 & 0
	}{
		0 & 1 & -1
	}{
		0 & -1 & 1
	}
\end{equation}
where $\rho$ is the fraction of migrations from the 0-jet bin that enter the \twojet bin. 
In this case, there are three nuisance parameters with uncertainty amplitudes
\begin{equation}
	\begin{array}{l@{}l@{}l}
		\begin{array}{r}
			\kappa^{\text{yield}}              : \Big( \\
			\kappa^{\text{mig}}_{0\rightarrow} : \Big( \\
			\kappa^{\text{mig}}_{1\rightarrow} : \Big(
		\end{array}
		&
		\begin{array}{@{}rrr@{}}
			\Delta^{\text{y}}_0, & \Delta^{\text{y}}_1, & \Delta^{\text{y}}_{\geq2} \\
			\Delta^{\text{mig}}_{0\rightarrow}, & -\parenths{1-\rho} \Delta^{\text{mig}}_{0\rightarrow}, & -\rho \Delta^{\text{mig}}_{0\rightarrow} \\
			0, & \Delta^{\text{mig}}_{1\rightarrow}, & -\Delta^{\text{mig}}_{1\rightarrow}
		\end{array}
		&
		\begin{array}{l}
			\Big) \\ \Big) \\ \Big) \,.
		\end{array}
	\end{array}
	\label{eq:ggF:3bin_np}
\end{equation}
So it is $\Delta^{\text{y}}_0$, $\Delta^{\text{y}}_1$, $\Delta^{\text{y}}_{\geq2}$, 
$\Delta^{\text{mig}}_{0\rightarrow}$, $\Delta^{\text{mig}}_{1\rightarrow}$ and $\rho$ 
that must be determined. Two different prescriptions for evaluating these shall now be 
examined.



\subsection{Combined inclusive prescription}
\label{sec:ggF:ci}

The \textit{combined inclusive} (CI) prescription\footnote{
	The CI prescription is also called the Stewart-Tackmann prescription, after its 
	original proponents.
} \cite{Stewart-Tackmann:2012} uses scale variations of $\sigma_{\geq1}$ and 
$\sigma_{\geq2}$ to probe the size of the higher order logarithmic corrections, and 
identifies that they are related to the bin migration uncertainties. It therefore chooses
\begin{equation}
	\text{CI:}
	\quad\quad \Delta^{\text{y}}_0 &= \Delta\sigma_{\total},&
	\quad\quad \Delta^{\text{y}}_1 &= 0,&
	\quad\quad \Delta^{\text{y}}_{\geq2} &= 0, \nonumber \\
	\quad\quad \Delta^{\text{mig}}_{0\rightarrow} &= \Delta\sigma_{\geq1},&
	\quad\quad \Delta^{\text{mig}}_{1\rightarrow} &= \Delta\sigma_{\geq2},&
	\quad\quad \rho &= 0
\end{equation}
where $\Delta\sigma_{\total}$, $\Delta\sigma_{\geq1}$ and $\Delta\sigma_{\geq2}$ are 
evaluated at fixed order by varying \mur and \muf.

This is equivalent to assuming inclusive cross sections have uncorrelated uncertainties
\begin{equation}
	\text{CI:} \quad\quad
	\sigma_N = \sigma_{\geq N} - \sigma_{\geq N+1}
	\quad\quad\Rightarrow\quad\quad
	\Delta\sigma_N^2 = \Delta\sigma_{\geq N}^2 + \Delta\sigma_{\geq N+1}^2 \,.
\end{equation}
Although this assumption is not believed to be exact, the CI prescription offers a 
practical solution to the cancellations described in \Section~\ref{sec:ggF:naive} (see 
\Figure~\ref{fig:ggF:ci}). It ensures that uncertainties in exclusive cross sections are 
larger than those in the corresponding inclusive cross section, \ie $\Delta\sigma_{N} > 
\Delta\sigma_{\geq N}$, and the large-\ptcut limit agrees with expectations. 

\begin{figure}[t]
	\includegraphics[width=0.495\textwidth]{tex/signal/sigma0_naive}
	\hfill
	\includegraphics[width=0.495\textwidth]{tex/signal/sigma0_CI}
	\caption{The exclusive 0-jet ggF cross section versus the jet \pt threshold 
	\cite{YR2}. The bands show the perturbative uncertainties evaluated using the 
	na\"{i}ve prescription (left) and the combined inclusive prescription (right).}
	\label{fig:ggF:ci}
\end{figure}

It should be emphasised that each inclusive cross section must be evaluated at the same 
order in \alphaS (\eg $\sigma_{\total}^{\text{NNLO}}$, $\sigma_{\geq1}^{\text{NLO}}$ and 
$\sigma_{\geq2}^{\text{LO}}$). This can restrict the application of the CI prescription. 
For example, an exclusive 2-jet bin cannot be added until $\sigma_{\total}$ is calculated 
at N$^3$LO in QCD. For the ggF contamination to the VBF signal region (defined with a 
central jet veto), $\sigma_{\geq2}^{\text{NLO}}$ and $\sigma_{\geq3}^{\text{LO}}$ can be 
used since they are evaluated in a significantly different phase space 
($\Delta\sigma_{\geq2}^{\text{NLO}}$ and $\Delta\sigma_{\geq2}^{\text{LO}}$ are treated as 
fully correlated).

\hnnlo \cite{HNNLO} is used to compute $\sigma_{\total}^{\text{NNLO}}$, 
$\sigma_{\geq1}^{\text{NLO}}$ and $\sigma_{\geq2}^{\text{LO}}$. However, the CI 
prescription can be improved by using the NNLO+NNLL(QCD)+NLO(EW) $\sigma_{\total}$ 
calculation \cite{YR3}, which has smaller perturbative uncertainties than 
$\sigma_{\total}^{\text{NNLO}}$. In order to combine these results whilst preserving the 
total normalisation, the jet bin fractions $f_N = \sigma_N/\sigma_{\total}$ from \hnnlo are 
used to propagate the $\Delta\sigma_{\geq N}$ to $\Delta\sigma_{N}$:
\begin{equation}
	\delta\sigma_0^2 &= \frac{1}{f_0^2}\delta\sigma_{\total}^2 + \parenths{\frac{1}{f_0}-1}^{\!\!2}\delta\sigma_{\geq1}^2 \label{eq:ggF:ci_1} \\
	\delta\sigma_1^2 &= \parenths{\frac{1-f_0}{f_1}}^{\!\!2}\delta\sigma_{\geq1}^2 + \parenths{\frac{1-f_0}{f_1}-1}^{\!\!2}\delta\sigma_{\geq2}^2 \label{eq:ggF:ci_2}
\end{equation}
where $\delta\sigma_i = \Delta\sigma_i/\sigma_i$. This assumes the uncertainties are 
Gaussian distributed, though the nuisance parameters are implemented as log-normal 
distributions in the fitting code. The $\Delta\sigma_{\geq N}$ are evaluated via 
independent variation of \mur and \muf in the range $\mH/4 \leq \mur,\muf \leq \mH$, whilst 
observing the constraint $1/2 \leq \mur/\muf \leq 2$. These are then propagated to 
$\Delta\sigma_N$ using (\ref{eq:ggF:ci_1}) and (\ref{eq:ggF:ci_2}), as shown in 
\Table~\ref{tab:ggF:ci}.

\begin{table}
	\begin{tabular}{r@{\hskip 0.3in}c@{\hskip 0.3in}r@{\hskip 0.3in}r}
		\toprule
		\multicolumn{1}{c@{\hskip 0.3in}}{$i$} & $f_i = \sigma_i/\sigma_{\total}$ & \multicolumn{1}{c@{\hskip 0.3in}}{$\sigma_i$ (\pico\barn)} & \multicolumn{1}{c}{$\Delta\sigma_i/\sigma_i$} \\
		\midrule
		$\geq\!0$ & --    & 19.27 &  7.8\% \\
		$\geq\!1$ & --    & \multicolumn{1}{c@{\hskip 0.3in}}{--} & 20.2\% \\
		$\geq\!2$ & --    & \multicolumn{1}{c@{\hskip 0.3in}}{--} & 69.7\% \\
		\midrule
		\cline{3-4}
		$0$       & 0.614 & \multicolumn{1}{|r@{\hskip 0.3in}}{11.83} & \multicolumn{1}{r|}{18.0\%} \\
		$1$       & 0.267 &  \multicolumn{1}{|r@{\hskip 0.3in}}{5.15} & \multicolumn{1}{r|}{42.6\%} \\
		$\geq\!2$ & --    &  \multicolumn{1}{|r@{\hskip 0.3in}}{2.29} & \multicolumn{1}{r|}{69.7\%} \\
		\cline{3-4}
		\bottomrule
	\end{tabular}
	\caption{Inputs and outputs (boxed) of the combined inclusive prescription for ggF, 
	with \unit{$\mH = 125$}{\GeV} and \unit{$\sqrt{s} = 8$}{\TeV}. All inputs computed by 
	\hnnlo except $\sigma_{\total}$ and $\Delta\sigma_{\total}$, which are from 
	\Reference~\cite{YR3}.}
	\label{tab:ggF:ci}
\end{table}



\subsection{Jet veto efficiency prescription}
\label{sec:ggF:jve}

The \textit{jet veto efficiency} (JVE) prescription \cite{JVE:NLL,JVE:NNLL} considers each 
exclusive cross section as a product of the total cross section and jet veto efficiencies
\begin{equation}
	\sigma_0 &= \sigma_{\total} \epsilon_0 \\
	\sigma_1 &= \sigma_{\total} \parenths{1-\epsilon_0} \epsilon_1 \\
	\sigma_{\geq2} &= \sigma_{\total} \parenths{1-\epsilon_0} \parenths{1-\epsilon_1}
\end{equation}
where $\epsilon_0$ and $\epsilon_1$ are the first and second jet veto efficiencies. That 
is, $\epsilon_0 = \epsilon_0\parenths{\ptcut}$ is the efficiency of a veto upon jets with 
$\pt > \ptcut$, and $\epsilon_1 = \epsilon_1\parenths{p_{\text{T}}^{\text{sel}}, \ptcut}$ 
is the efficiency of a veto upon additional jets with $\pt > \ptcut$ given that there is 
already a jet with $\pt > p_{\text{T}}^{\text{sel}}$. In the \HWW analysis, 
$\ptcut = p_{\text{T}}^{\text{sel}} = \unit{25}{\GeV}$.\footnote{
	Since the jets are predominantly central, the raised \ptcut of \unit{30}{\GeV} in the 
	forward region is neglected. This approximation is conservative, since the 
	$\Delta\sigma_N$ decrease with higher \ptcut.
}

The JVE prescription assumes that uncertainties in $\sigma_{\total}$ and the $\epsilon_N$ 
are uncorrelated, which ensures that uncertainties in exclusive cross section are larger 
than those in the total cross section, \ie $\Delta\sigma_N > \Delta\sigma_{\total}$.
This is equivalent to choosing
\begin{equation}
	\text{JVE:}
	\quad\quad \Delta^{\text{y}}_0 &= \Delta\sigma_{\total} \, \frac{\sigma_{0}}{\sigma_{\total}} \,,&
	\quad\quad \Delta^{\text{y}}_1 &= \Delta\sigma_{\total} \, \frac{\sigma_{1}}{\sigma_{\total}} \,,&
	\quad\quad \Delta^{\text{y}}_{\geq2} &= \Delta\sigma_{\total} \, \frac{\sigma_{\geq2}}{\sigma_{\total}} \,, \nonumber \\
	\quad\quad \Delta^{\text{mig}}_{0\rightarrow} &= \Delta\epsilon_0 \, \sigma_{\total} \,,&
	\quad\quad \Delta^{\text{mig}}_{1\rightarrow} &= \Delta\epsilon_1 \, \sigma_{\geq1} \,,&
	\quad\quad \rho &= 1-\epsilon_1 \,. \label{eq:jve}
\end{equation}
Thus, the JVE prescription requires six inputs: $\sigma_{\total}$, $\epsilon_0$, 
$\epsilon_1$, $\Delta\sigma_{\total}$, $\Delta\epsilon_0$, $\Delta\epsilon_1$.
As in the CI prescription, $\sigma_{\total}$ and $\Delta\sigma_{\total}$ are taken from the 
NNLO+NNLL(QCD)+NLO(EW) calculation. However, the $\epsilon_N$ contain similar cancellations 
to those discussed in \Section~\ref{sec:ggF:naive}, and so the $\Delta\epsilon_N$ must be 
treated with care.

There is an ambiguity in the definition of $\epsilon_N$ that is not present in fixed order 
$\sigma_{\geq N}$ calculations \cite{JVE:NLL}. For example, considering NNLO terms with 
respect to the $N$-jet process, three alternative definitions of $\epsilon_N$ may be 
identified
\begin{equation}
	\epsilon_N^{\parenths{\text{a}}} &= 1 - \frac{\sigma_{\geq N+1}^{\text{NLO}}}{\sigma_{\geq N}^{\text{NNLO}}} \label{eq:jve:scheme_a} \\
	\epsilon_N^{\parenths{\text{b}}} &= 1 - \frac{\sigma_{\geq N+1}^{\text{NLO}}}{\sigma_{\geq N}^{\text{NLO}}} \label{eq:jve:scheme_b} \\
	\epsilon_N^{\parenths{\text{c}}} &= 1 - \frac{\sigma_{\geq N+1}^{\text{NLO}}}{\sigma_{\geq N}^{\text{LO}}} + \parenths{\frac{\sigma_{\geq N}^{\text{NLO}}}{\sigma_{\geq N}^{\text{LO}}} - 1} \frac{\sigma_{\geq N+1}^{\text{LO}}}{\sigma_{\geq N}^{\text{LO}}} \label{eq:jve:scheme_c} \,.
\end{equation}
Although scheme (a) is the more intuitive definition, schemes (b) and (c) differ by N$^3$LO 
terms and therefore probe higher order corrections.\footnote{
	Processes whose perturbative series converge more quickly, such as 
	\HepProcess{\Pquark\APquark \HepTo \PZ}, exhibit better agreement between schemes.
} 
However, this probing is less susceptible to accidental cancellations than scale variations.
Thus, $\epsilon_N^{\parenths{\text{a}}}$ is used as the nominal $\epsilon_N$, whilst 
$\Delta\epsilon_N$ is evaluated via scale variations in $\epsilon_N^{\parenths{\text{a}}}$ 
or the difference between the schemes, whichever is larger.

\begin{figure}[t]
	\includegraphics[width=0.495\textwidth]{tex/signal/eps0_jve_fixedorder}
	\hfill
	\includegraphics[width=0.495\textwidth]{tex/signal/eps0_jve_resummed}
	\caption{Jet veto efficiency $\epsilon_0$ versus the jet \pt threshold, computed with 
	fixed order (left) and resummed (right) calculations by \jetvheto \cite{JVE:NNLL}. The 
	bands show scale uncertainties. The bands of schemes (b) and (c) are not used in 
	$\Delta\epsilon_0$.}
	\label{fig:ggF:jve_eps1}
\end{figure}

\Figure~\ref{fig:ggF:jve_eps1} shows how scheme differences of $\epsilon_0$ inflate the 
perturbative uncertainties compared to scale variations of 
$\epsilon_0^{\parenths{\text{a}}}$. At \unit{$\ptcut = 25$}{\GeV}, it increases 
$\Delta\epsilon_0$ from \about5\% to \about20\%. \Figure~\ref{fig:ggF:jve_eps1} also shows 
how resummation of the $\ln\parenths{\ptcut/\mH}$ logarithms to all orders of \alphaS can 
improve the estimation of $\epsilon_0$, resulting in better agreement between schemes and 
consequently reducing $\Delta\epsilon_0$. This resummation includes NNLL terms and is 
performed by \jetvheto \cite{JVE:NNLL}.

The three NNLO schemes (\ref{eq:jve:scheme_a}), (\ref{eq:jve:scheme_b}) and 
(\ref{eq:jve:scheme_c}) can also be used to define $\epsilon_1$. This offers an improvement 
compared to the CI prescription, which is currently limited to using 
$\sigma_{\geq1}^{\text{NLO}}$ and $\sigma_{\geq2}^{\text{LO}}$ (see 
\Section~\ref{sec:ggF:ci}). Unfortunately, it is not possible to calculate 
$\epsilon_1^{\text{(a)}}$ since a full $\sigma_{\geq1}^{\text{NNLO}}$ calculation is not 
yet available. Instead, we choose $\epsilon_1 = (\epsilon_1^{\text{(b)}} + 
\epsilon_1^{\text{(c)}})/2$ and $\Delta\epsilon_1$ is evaluated by an envelope of all scale 
uncertainties in $\epsilon_1^{\text{(b)}}$ and $\epsilon_1^{\text{(c)}}$.
The validity of this approximation is tested using \HepProcess{\Pgluon\Pgluon}-initiated 
diagrams only, for which a $\sigma_{\geq1}^{\text{NNLO}}$ calculation exists 
\cite{H+1j:NNLO}. For $k_{\text{T}}$ jets with $R = 0.5$ and \unit{$\ptcut = 30$}{\GeV}, we 
find that $\epsilon_1^{\text{(a)}} = 0.831$, $\epsilon_1^{\text{(b)}} = 0.761$ and 
$\epsilon_1^{\text{(c)}} = 0.843$.

The NNLO+NNLL(QCD)+NLO(EW) $\sigma_{\total}$ calculation, the \jetvheto $\epsilon_0$ 
calculation and the \mcfm $\epsilon_1$ calculation are used as inputs to the JVE 
prescription (\ref{eq:jve}). \Table~\ref{tab:ggF:jve} shows the jet-binned cross sections 
and uncertainties obtained using simple Gaussian propagation of uncertainties.

\begin{table}[t]
	\begin{tabular}{l@{\hskip 0.3in}r@{$\,\pm\,$}l}
		\toprule
		$\sigma_{\total}$ (\pico\barn) & 19.27 & 1.50 \\
		$\epsilon_0$                   & 0.613 & 0.072 \\
		$\epsilon_1$                   & 0.615 & 0.061 \\
		\midrule
		$\sigma_0$ (\pico\barn)        & 11.81 & 1.66 \\
		$\sigma_1$ (\pico\barn)        &  4.59 & 1.03 \\
		$\sigma_{\geq2}$ (\pico\barn)  &  2.87 & 0.73 \\
		\bottomrule
	\end{tabular}
	\caption{Results of the jet veto efficiency prescription for ggF, with 
	\unit{$\mH = 125$}{\GeV} and \unit{$\sqrt{s} = 8$}{\TeV}. $\sigma_{\total}$ is from 
	\Reference~\cite{YR3}, $\epsilon_0$ is from \jetvheto and $\epsilon_1$ is from \mcfm.}
	\label{tab:ggF:jve}
\end{table}

\Figure~\ref{fig:ggF:jve_compare} compares the $\epsilon_0$ and $\epsilon_1$ calculations 
described above to a variety of different \meps{\powhegbox}{\pythia{8}} configurations.
The $\epsilon_N$ calculations were performed at parton-level in the large-$m_{\Ptop}$ 
limit, and the solid lines show similar configurations of \meps{\powhegbox}{\pythia{8}}, 
and can be considered directly comparable. The effect of consecutively adding 
hadronisation, MPI and finite quark mass effects are also shown. Finally, the blue line 
shows how reweighting the Higgs boson \pt distribution (see \Section~\ref{sec:ggF:pt}) 
influences these observables. \Figure~\ref{fig:ggF:jve_compare} shows that, following 
\ptH reweighting, the MC is within the perturbative uncertainty band.

\begin{figure}[t]
	\includegraphics[width=0.495\textwidth]{tex/signal/eps0_jve_compare}
	\hfill
	\includegraphics[width=0.495\textwidth]{tex/signal/eps1_jve_compare}
	\caption{Veto efficiencies of a first jet (left) and a second jet (right), versus the 
	jet \pt threshold. Fixed order (green) and resummed (red) results are shown with their 
	perturbative uncertainties, and are compared to different MC scenarios.}
	\label{fig:ggF:jve_compare}
\end{figure}





\subsection{Discussion of results}

It is helpful to directly compare the predicted jet-binned cross sections of the two 
prescriptions, as in \Figure~\ref{fig:signal:jetbin_xs_summary}. ``Fixed order CI'' and 
``resummed JVE'' are the prescriptions described in the preceding sections. JVE offers a 
reduction in uncertainty compared to CI, whilst both prescriptions remain consistent with 
the \meps{\powhegbox}{\pythia{8}} MC used, both before and after \ptH reweighting.

``Fixed order JVE'' replaces the resummed $\epsilon_0$ calculation with the fixed order 
$\epsilon_0$ calculation (see \Figure~\ref{fig:ggF:jve_compare}). This uses the same 
degree of accuracy as ``fixed order CI'' in calculating $\sigma_0$, and so their comparison 
in the the 0-jet bin directly evaluates how conservative each prescription is. JVE appears 
more conservative than CI.

\todo[inline]{Add sentence about which prescription is chosen}

\begin{figure}[t]
	\includegraphics[width=\largefigwidth]{custom_images/ggF_xs_jetbin}
	\caption{Jet-binned cross sections for ggF production with \unit{$\mH = 125$}{\GeV} and 
	\unit{$\sqrt{s} = 8$}{\TeV}. The error bars show the perturbative uncertainty 
	associated with each prescription. Consistency with \meps{\powhegbox}{\pythia{8}} is also shown, normalised to \unit{19.27}{\pico\barn}.}
	\label{fig:signal:jetbin_xs_summary}
\end{figure}
