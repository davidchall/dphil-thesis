%!TEX root = ../../thesis.tex

\ac{ggF} is modelled by \meps{\powhegbox}{\pythia{8}}, including the exact mass 
dependence of the \Ptop and \Pbottom quarks in the loop \cite{Powheg-ggF-quarkmasses}. 
The CT10 PDF \cite{CTEQ} was used to model the incoming partons.
The \powhegbox parameter \verb|hfact| controls the scale at which the emission 
transitions from Sudakov-like to ME-like. This is tuned to $\mH/1.2$ in order to 
reproduce the Higgs boson \pt distribution of \hqt \cite{HqT2} (NLO+NNLL accuracy). This 
tuning is further discussed in \Section~4.9 of \Reference \cite{YR2}.



\subsection{Higgs boson transverse momentum}
\todo[inline]{Include Higgs \pt studies?}



\subsection{Event selection acceptance}

In \Section~\ref{sec:ggf_jetbin}, perturbative uncertainties in the jet binning were 
considered. We now consider uncertainties in the acceptance of the event selection. These 
are evaluated at hadron-level (\ie before detector simulation) by changing some aspect of 
the MC modelling and measuring the effect upon the acceptance within each jet bin. 
The selection criteria are similar to those applied at detector-level (see 
\Table~\ref{tab:signal:acc_truthselection}).

Hadron-level object definitions follow. The MC event record is used to identify leptons 
and neutrinos which descend from the Higgs boson. An \metvec vector is constructed from 
the neutrinos. Each lepton is `dressed' with the four-momenta of photons within a cone of 
$\Delta R < 0.1$, in order to recover energy lost via QED FSR. Jets are found using 
individual particles as inputs (\cf topo-clusters at detector-level). Muons and neutrinos 
are excluded from jet finding since they interact weakly with the calorimeter. Objects 
must pass the same \pt, $\eta$ and overlap removal criteria applied at detector-level.

\begin{table}
	\begin{tabular}{ccc}
		\toprule
		Jet binning & \emch/\mech & \eech/\mmch \\
		\midrule
		Inclusive & \multicolumn{2}{c}{$\ptleadlep > 22$ and $\ptsubleadlep > 10$} \\
		& $\mll > 10$ & $\mll > 12$ \\
		& -- & $\mods{\mll - \mZ} > 15$ \\
		& $\met > 20$ & $\metrel > 40$ \\
		& \multicolumn{2}{c}{$\mll < 55$} \\
		& \multicolumn{2}{c}{$\dphill < 1.8$} \\
		\midrule
		0-jet & \multicolumn{2}{c}{$\dphillmet > \pi/2$} \\
		& \multicolumn{2}{c}{$\ptll > 30$} \\
		\midrule
		1-jet & $\maxmtw > 50$ & -- \\
		& $\mtautau < \mZ - 25$ & -- \\
		\midrule
		\twojet & $\mtautau < \mZ - 25$ & -- \\
		& \multicolumn{2}{c}{Fail $\dyjj > 3.6$ or $\mjj > 600$ or CJV or OLV} \\
		\bottomrule
	\end{tabular}
	\caption{Hadron-level event selection used to calculate ggF acceptance uncertainties. 
	Cuts on energy, momentum and mass are given in \GeV, and angular cuts are given in 
	radians. The CJV and OLV are the central jet veto and outside lepton veto, 
	respectively. See \Chapter~\ref{chap:selection} for a detailed explanation of the 
	criteria.}
	\label{tab:signal:acc_truthselection}
\end{table}

Four sources of theoretical uncertainty are considered:
\begin{itemize}[noitemsep,nolistsep]
	\item higher order corrections,
	\item \acp{PDF},
	\item parton shower, hadronisation and underlying event models,
	\item NLO-PS matching scheme.
\end{itemize}

Uncertainties due to higher order corrections are evaluated via independent variation of 
renormalisation and factorisation scales in the range $\mH/2 \leq \mur,\muf \leq 2\mH$, 
whilst observing the constraint $1/2 \leq \mur/\muf \leq 2$. In the \twojet bin, scale 
uncertainties are evaluated \todo{H+2j scale uncertainties in acceptance} with \mcfm 
\cite{MCFM:H2j}. This is necessary because \powhegbox is an NLO generator and therefore 
cannot probe perturbative uncertainties in the \twojet bin.

Uncertainties due to \acp{PDF} are evaluated in two ways. The acceptance is compared to 
that predicted with the MSTW2008 PDF \cite{MSTW}. Also, the set of PDF eigenvectors 
corresponding to 90\% \ac{CL} of the CT10 fit were used to evaluate an uncertainty, 
which was then rescaled to 68\% \ac{CL}. PDF uncertainties are calculated relative to the 
inclusive cross section, in order to include uncertainties in the jet binning.

Uncertainties due to the \ac{PS}, hadronisation and \ac{UE} models are evaluated by 
comparing \powhegbox showered by \pythia{8} (nominal), \pythia{6} and \fherwig. 
Uncertainties due to the NLO-PS matching scheme are evaluated by comparing 
\meps{\powhegbox}{\fherwig} to \meps{\mcatnlo}{\herwigpp}.

The acceptance uncertainties for each signal region used in the fitting procedure are 
shown in \Table~\ref{tab:signal:acc_unc}.

\begin{table}
	\centering
	\begin{tabular}{ccc|cccccc}
		\toprule
		& \mll & \ptsubleadlep & \multirow{2}{*}{Scale} & \multicolumn{2}{c}{PDF} & \multicolumn{2}{c}{PS/Had./UE} & \multirow{2}{*}{NLO-PS} \\
		& (\GeV) & (\GeV) & & MSTW & 68\% CL & \pythia{6} & \fherwig & \\
		\midrule
		\multicolumn{9}{c}{\eech/\mmch channels} \\
		\midrule
		0-jet & 12--55 & $>10$ & 1.4\% & +1.9\% & 3.2\% &   $+1.6\%$ & $+6.4\%$ &   $-2.5\%$ \\
		1-jet & 12--55 & $>10$ & 1.9\% & +1.8\% & 2.8\% & $(-)1.5\%$ & $+2.1\%$ & $(-)1.4\%$ \\
		\midrule
		\multicolumn{9}{c}{\emch/\mech channels} \\
		\midrule
		\multirow{6}{*}{0-jet}
		& \multirow{3}{*}{10--30}
	    &  10--15 & 2.6\% & +1.8\% & 3.2\% &   $-1.7\%$ &   $+5.7\%$ &   $-3.5\%$ \\
		&& 15--20 & 1.3\% & +1.9\% & 3.2\% & $(+)2.4\%$ &   $+4.9\%$ &   $-2.9\%$ \\
		&&  $>20$ & 1.0\% & +1.9\% & 3.2\% &   $-2.2\%$ & $(-)1.6\%$ & $(-)1.4\%$ \\
		\cmidrule(lr){2-9}
		& \multirow{3}{*}{30--55}
		&  10--15 & 1.5\% & +1.8\% & 3.3\% & $(+)2.0\%$ &   $+5.5\%$ &   $-3.8\%$ \\
		&& 15--20 & 1.5\% & +1.9\% & 3.3\% & $(-)2.5\%$ & $(+)2.4\%$ &   $-2.5\%$ \\
		&&  $>20$ & 3.5\% & +1.9\% & 3.3\% &   $-1.9\%$ &   $-2.4\%$ & $(-)1.3\%$ \\
		\cmidrule(lr){1-9}
		\multirow{6}{*}{1-jet}
		& \multirow{3}{*}{10--30}
	    &  10--15 & 3.2\% & +1.7\% & 2.9\% &   $+2.9\%$ &  $+10.8\%$ &   $-3.8\%$ \\
		&& 15--20 & 2.9\% & +1.8\% & 2.9\% & $(+)3.8\%$ & $(+)3.9\%$ & $(+)3.6\%$ \\
		&&  $>20$ & 3.5\% & +1.8\% & 2.7\% & $(+)2.1\%$ & $(+)2.0\%$ & $(-)1.9\%$ \\
		\cmidrule(lr){2-9}
		& \multirow{3}{*}{30--55}
		&  10--15 & 5.8\% & +1.7\% & 3.0\% & $(+)3.2\%$ &  $+11.4\%$ &   $-6.8\%$ \\
		&& 15--20 & 1.0\% & +1.8\% & 3.3\% & $(+)2.6\%$ &  $+13.5\%$ &   $+6.7\%$ \\
		&&  $>20$ & 1.3\% & +1.8\% & 2.8\% & $(-)1.9\%$ & $(-)1.8\%$ & $(+)1.7\%$ \\
		\cmidrule(lr){1-9}
		\twojet & 10--55 & $>10$ &  18\% & +2.0\% & 2.2\% & $(-)1.7\%$ & $(+)1.7\%$ & $-4.5\%$ \\
		\bottomrule
	\end{tabular}
	\caption{Theoretical uncertainties in the ggF acceptance for each signal region used 
	in the fitting procedure. PDF uncertainties are relative to the inclusive cross 
	section, whereas others are calculated within jet bins. When the uncertainty is 
	statistically insignificant, the statistical uncertainty on the generator difference 
	is given, and the sign of the generator difference is parenthesised.}
	\label{tab:signal:acc_unc}
\end{table}



\subsection{\mt shape modelling}

Theoretical uncertainties in the shape of the \mt distribution were also investigated, 
as they could have an impact on the fit. Scale uncertainties are presented in 
\Figure~\ref{fig:signal:mT_scale}, and \Figure~\ref{fig:signal:mT_psue} shows those due 
to the PS/hadronisation/UE model and NLO-PS matching. They are driven by differences 
between \pythia{8} and \fherwig, though remain small.

\begin{figure}
	\includegraphics[width=0.495\textwidth]{tex/signal/mT_0jet_scale}
	\hfill
	\includegraphics[width=0.495\textwidth]{tex/signal/mT_1jet_scale}
	\caption{Scale uncertainties in the shape of the ggF \mt distribution, for the 0-jet 
	(left) and 1-jet (right) signal regions. The nominal scale is $\mu_0 = \mH$.}
	\label{fig:signal:mT_scale}
\end{figure}

\begin{figure}
	\includegraphics[width=0.495\textwidth]{tex/signal/mT_0jet_psue}
	\hfill
	\includegraphics[width=0.495\textwidth]{tex/signal/mT_1jet_psue}
	\caption{Uncertainties due to the PS/hadronisation/UE model and NLO-PS matching scheme
	in the shape of the ggF \mt distribution, for the 0-jet (left) and 1-jet (right) 
	signal regions. When unstated, the matrix element is provided by \powhegbox. 
	NLO-PS uncertainties are evaluated by comparing the red and blue lines.}
	\label{fig:signal:mT_psue}
\end{figure}



\subsection{ME-PS matching}

When studying the uncertainties described above, a discrepancy was observed at high jet 
multiplicity between \meps{\powhegbox}{\pythia{8}} and \meps{\powhegbox}{\pythia{6}} (see 
green and black lines in \Figure~\ref{fig:signal:matching}). Its effect is properly 
accounted for in the above uncertainties, but the discrepancy is interesting in itself.
It was also observed in other electroweak processes.

The hadronisation and \ac{UE} models of standalone \pythia{8} have been tuned to ATLAS 
\ac{UE} data with a variety of \ac{PDF} sets (known as AU2 tunes) \cite{ATLAS:tune:2012}.
However, the parton shower was not tuned since the default settings gave good agreement. 

When modelling ggF with \powhegbox, the AU2-CT10 tune was used in order to match the 
\acp{PDF} used in the matrix element calculation. Technically speaking, a dedicated 
\meps{\powhegbox}{\pythia{8}} tune should have been used, but this was unavailable. 
Unfortunately, a couple of issues had a negative impact on the NLO-PS matching. First, 
the parton shower evolves \alphaS at LO while NLO \acp{PDF} were used in the shower. 
Second, there was a mismatch between the \alphaS used in \powhegbox, 
$\alphaS\parenths{\mZ} = 0.118$, and the default value in the parton shower, 
$\alphaS\parenths{\mZ} = 0.137$. The effect of these issues is shown in 
\Figure~\ref{fig:signal:matching}.

\begin{figure}
	\includegraphics[width=\smallfigwidth]{tex/signal/matching}
	\caption{Jet multiplicity produced by \meps{\powhegbox}{\pythia{8}} with a selection 
	of shower tunes. The green circles is the tune used in the analysis. The red squares 
	change the parton shower PDFs from CT10 to CTEQ6L1. The blue triangles additionally 
	change the parton shower $\alphaS\parenths{\mZ}$ from 0.137 to 0.118 (in agreement 
	with \powhegbox). \meps{\powhegbox}{\pythia{6}} is shown in black for reference.}
	\label{fig:signal:matching}
\end{figure}

Identification of this poor matching has led to improvements in the latest round of MC 
tuning, where dedicated \meps{\powhegbox}{\pythia{8}} tunes are fit using an adjusted 
parton shower \cite{ATLAS:tune:2013}.
