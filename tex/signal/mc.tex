%!TEX root = ../../thesis.tex

The ggF signal is modelled by \meps{\powhegbox}{\pythia{8}}, including the exact mass 
dependence of the \Ptop and \Pbottom quarks in the loop \cite{Powheg-ggF-quarkmasses}, and 
using the CT10 PDF \cite{CTEQ} to describe the incoming partons. Various aspects of this 
modelling are now discussed.



\subsection{Higgs boson transverse momentum}
\label{sec:ggF:pt}

The MC events are produced with the \powhegbox parameter \verb|hfact| tuned to 
$\mH/1.2$.\footnote{
	\texttt{hfact} controls the scale at which the first emission transitions from 
	Sudakov-like to ME-like. Its use in tuning \ptH is discussed in Sections~4.5 and 4.9 of 
	\Reference~\cite{YR2}.
}
This setting reproduces the NLO+NNLL \ptH distribution calculated by \hqt \cite{HqT2} in 
the large-$m_{\Ptop}$ limit, when effects of hadronisation, MPI and finite quark masses are 
turned off in \meps{\powhegbox}{\pythia{8}}.

The \ptH distribution of the generated MC events is reweighted to the best available 
prediction of \ptH.\footnote{
	The reweighting is motivated by the \HepProcess{\PHiggs \HepTo \Pphoton\Pphoton} and 
	\HepProcess{\PHiggs \HepTo \Ptau\Ptau} analyses, which feature boosted categories.
}
This reweighting simultaneously satisfies three criteria:
\begin{itemize}[noitemsep,nolistsep]
	\item inclusive \ptH distribution agrees with \hres \cite{HRes}
	\item \twojet \ptH distribution agrees with \HepProcess{\Pgluon\Pgluon \HepTo 
	\PHiggs jj} process of \meps{\minlo}{\pythia{8}} \cite{Minlo:Hjj}
	\item jet-binned cross sections agree with \Section~\ref{sec:ggf_jetbin}.
\end{itemize}
\hres computes the inclusive \ptH distribution with NLO+NNLL accuracy and includes finite 
$m_{\Ptop}$ and $m_{\Pbottom}$ effects in the loop. It also employs a dynamic scale of 
$\mu_0 = \sqrt{m_{\PHiggs}^2 + p_{\text{T,\PHiggs}}^2}$ as the nominal \mur and \muf scale, 
which improves results at high \ptH compared to the fixed $\mu_0 = \mH$ scale used by \hqt.
\minlo is an improved version of \powhegbox, which includes higher order logarithmic 
contributions through the careful choice of \mur and \muf scales \cite{Minlo:Hjj}. 
\Figure~\ref{fig:signal:jetbin_xs_summary} shows that the reweighting preserves agreement 
with the predicted \njets distribution.



\subsection{Event selection acceptance}
\label{sec:ggF:acc}

In \Section~\ref{sec:ggf_jetbin}, perturbative uncertainties in the jet binning were 
considered. We now consider uncertainties in the acceptance of the remaining event 
selection criteria (and PDF uncertainties in the jet binning). These 
are evaluated at hadron-level (\ie before detector simulation) by changing some aspect of 
the MC modelling and measuring the effect upon the acceptance relative to the jet-binned 
cross sections (in the case of PDF uncertainties, the acceptance is calculated relative to 
the inclusive cross section). The selection criteria are similar to those applied at 
detector-level (see \Table~\ref{tab:signal:acc_truthselection}).

Hadron-level object definitions follow. The MC event record is used to identify leptons 
and neutrinos which descend from the Higgs boson. An \metvec vector is constructed from 
the neutrinos. Each lepton is `dressed' with the four-momenta of photons within a cone of 
$\Delta R < 0.1$, in order to recover energy lost via QED FSR. Jets are found using 
individual particles as inputs (\cf topo-clusters at detector-level). Muons and neutrinos 
are excluded from jet finding since they interact weakly with the calorimeter. Objects 
must pass the same \pt, $\eta$ and overlap removal criteria applied at detector-level.

\begin{table}[t]
	\begin{tabular}{ccc}
		\toprule
		Jet binning & \emch/\mech & \eech/\mmch \\
		\midrule
		Inclusive & \multicolumn{2}{c}{$\ptleadlep > 22$ and $\ptsubleadlep > 10$} \\
		& $\mll > 10$ & $\mll > 12$ \\
		& -- & $\mods{\mll - \mZ} > 15$ \\
		& $\met > 20$ & $\metrel > 40$ \\
		& \multicolumn{2}{c}{$\mll < 55$} \\
		& \multicolumn{2}{c}{$\dphill < 1.8$} \\
		\midrule
		0-jet & \multicolumn{2}{c}{$\dphillmet > \pi/2$} \\
		& \multicolumn{2}{c}{$\ptll > 30$} \\
		\midrule
		1-jet & $\maxmtw > 50$ & -- \\
		& $\mtautau < \mZ - 25$ & -- \\
		\midrule
		\twojet & $\mtautau < \mZ - 25$ & -- \\
		& \multicolumn{2}{c}{Fail $\dyjj > 3.6$ or $\mjj > 600$ or CJV or OLV} \\
		\bottomrule
	\end{tabular}
	\caption{Hadron-level event selection criteria used to calculate ggF acceptance 
	uncertainties. Cuts on energy, momentum and mass are given in \GeV, and angular cuts 
	are given in radians. The CJV and OLV are the central jet veto and outside lepton veto, 
	respectively. See \Chapter~\ref{chap:selection} for a detailed explanation of the 
	criteria.}
	\label{tab:signal:acc_truthselection}
\end{table}

\newpage
Four sources of theoretical uncertainty are considered:
\begin{itemize}[noitemsep,nolistsep]
	\item higher order corrections,
	\item PDFs,
	\item parton shower, hadronisation and underlying event models,
	\item NLO-PS matching scheme.
\end{itemize}

Uncertainties due to higher order corrections are evaluated via independent variation of 
renormalisation and factorisation scales in the range $\mH/2 \leq \mur,\muf \leq 2\mH$, 
whilst observing the constraint $1/2 \leq \mur/\muf \leq 2$. In the \twojet bin, scale 
uncertainties are evaluated \todo{H+2j scale uncertainties in acceptance} with \mcfm 
\cite{MCFM:H2j}. This is necessary because \powhegbox is an NLO generator and therefore 
cannot probe perturbative uncertainties in the \twojet bin.

Uncertainties due to PDFs are evaluated in two ways. First, the acceptance is compared to 
that predicted with the MSTW2008 PDF \cite{MSTW}. Second, the set of PDF eigenvectors 
corresponding to 90\% CL of the CT10 fit were used to evaluate an uncertainty, 
which was then rescaled to 68\% CL. PDF uncertainties are evaluated using \mcatnlo.

Uncertainties due to the parton shower (PS), hadronisation and underlying event (UE) 
models are evaluated by comparing \powhegbox showered by \pythia{8} (nominal), \pythia{6} 
and \fherwig. Uncertainties due to the NLO-PS matching scheme are evaluated by comparing 
\meps{\powhegbox}{\fherwig} to \meps{\mcatnlo}{\herwigpp}.

\Table~\ref{tab:signal:acc_unc} shows the acceptance uncertainties for each signal region 
used in the fitting procedure, including the \ptsubleadlep and \mll split signal regions 
in the \emch/\mech channels.

\begin{table}[p]
	\centering
	\begin{tabular}{ccc|cccccc}
		\toprule
		& \mll & \ptsubleadlep & \multirow{2}{*}{Scale} & \multicolumn{2}{c}{PDF} & \multicolumn{2}{c}{PS/UE} & \multirow{2}{*}{NLO-PS} \\
		& (\GeV) & (\GeV) & & collab. & 68\% CL & \pythia{6} & \fherwig & \\
		\midrule
		\multicolumn{9}{c}{\eech/\mmch channels} \\
		\midrule
		0-jet & 12--55 & $>10$ & 1.4\% & +1.9\% & 3.2\% &   $+1.6\%$ & $+6.4\%$ &   $-2.5\%$ \\
		1-jet & 12--55 & $>10$ & 1.9\% & +1.8\% & 2.8\% & $(-)1.5\%$ & $+2.1\%$ & $(-)1.4\%$ \\
		\midrule
		\multicolumn{9}{c}{\emch/\mech channels} \\
		\midrule
		\multirow{6}{*}{0-jet}
		& \multirow{3}{*}{10--30}
	    &  10--15 & 2.6\% & +1.8\% & 3.2\% &   $-1.7\%$ &   $+5.7\%$ &   $-3.5\%$ \\
		&& 15--20 & 1.3\% & +1.9\% & 3.2\% & $(+)2.4\%$ &   $+4.9\%$ &   $-2.9\%$ \\
		&&  $>20$ & 1.0\% & +1.9\% & 3.2\% &   $-2.2\%$ & $(-)1.6\%$ & $(-)1.4\%$ \\
		\cmidrule(lr){2-9}
		& \multirow{3}{*}{30--55}
		&  10--15 & 1.5\% & +1.8\% & 3.3\% & $(+)2.0\%$ &   $+5.5\%$ &   $-3.8\%$ \\
		&& 15--20 & 1.5\% & +1.9\% & 3.3\% & $(-)2.5\%$ & $(+)2.4\%$ &   $-2.5\%$ \\
		&&  $>20$ & 3.5\% & +1.9\% & 3.3\% &   $-1.9\%$ &   $-2.4\%$ & $(-)1.3\%$ \\
		\cmidrule(lr){1-9}
		\multirow{6}{*}{1-jet}
		& \multirow{3}{*}{10--30}
	    &  10--15 & 3.2\% & +1.7\% & 2.9\% &   $+2.9\%$ &  $+10.8\%$ &   $-3.8\%$ \\
		&& 15--20 & 2.9\% & +1.8\% & 2.9\% & $(+)3.8\%$ & $(+)3.9\%$ & $(+)3.6\%$ \\
		&&  $>20$ & 3.5\% & +1.8\% & 2.7\% & $(+)2.1\%$ & $(+)2.0\%$ & $(-)1.9\%$ \\
		\cmidrule(lr){2-9}
		& \multirow{3}{*}{30--55}
		&  10--15 & 5.8\% & +1.7\% & 3.0\% & $(+)3.2\%$ &  $+11.4\%$ &   $-6.8\%$ \\
		&& 15--20 & 1.0\% & +1.8\% & 3.3\% & $(+)2.6\%$ &  $+13.5\%$ &   $+6.7\%$ \\
		&&  $>20$ & 1.3\% & +1.8\% & 2.8\% & $(-)1.9\%$ & $(-)1.8\%$ & $(+)1.7\%$ \\
		\cmidrule(lr){1-9}
		\twojet & 10--55 & $>10$ &  18\% & +2.0\% & 2.2\% & $(-)1.7\%$ & $(+)1.7\%$ & $-4.5\%$ \\
		\bottomrule
	\end{tabular}
	\caption{Theoretical uncertainties in the ggF acceptance for each signal region used 
	in the fitting procedure. PDF uncertainties are in acceptances relative to the 
	inclusive cross section, whereas others are calculated within jet bins. When the 
	uncertainty is statistically insignificant, the statistical uncertainty on the 
	generator difference is given, and the sign of the generator difference is 
	parenthesised.}
	\label{tab:signal:acc_unc}
\end{table}



\subsection{\mt shape modelling}
\label{sec:ggF:mt}

Theoretical uncertainties in the shape of the \mt distribution are also investigated. 
Uncertainties due to scale, PS/UE and NLO-PS choices are 
considered using the methods described above. The split signal regions are not used in 
this study since the statistical fluctuations in the \mt distributions are large.

Each uncertainty is parametrised by fitting the ratio of the \mt shapes, and then 
symmetrising the fit to produce ``up'' and ``down'' variations. The \mt distributions are 
normalised to unit integral in order to remove effects from acceptance uncertainties. 
In cases where multiple variations exist within a single uncertainty source (such as the 
seven scale variations), the largest deviation from the nominal result is fit. A linear 
fit is used in the central \mt region, and a constant is used in the low-\mt and 
high-\mt tails of the distribution where statistical fluctuations dominate.

These fits allow the hadron-level \mt distribution of the ggF signal to be reweighted 
to the ``up'' and ``down'' variations. In this way, the \mt shape uncertainty is treated 
as a nuisance parameter in the \HWW fitting procedure. The uncertainties for the 0-jet 
and 1-jet signal regions are displayed in \Figure~\ref{fig:signal:mTshape}.

\begin{figure}[t]
	\includegraphics[width=\largefigwidth]{custom_images/mT-shapes/ggf}
	\caption{ggF \mt shape systematic uncertainties in the 0-jet and 1-jet signal regions 
	of the \emch/\mech channels.}
	\label{fig:signal:mTshape}
\end{figure}



\subsection{ME-PS matching}
\label{sec:ggF:meps_matching}

When studying the uncertainties described above, a discrepancy was observed at high jet 
multiplicity between \meps{\powhegbox}{\pythia{8}} and \meps{\powhegbox}{\pythia{6}} (see 
the green and black lines in \Figure~\ref{fig:signal:matching}). Although its effect is 
included in the acceptance uncertainties, it is intrinsically interesting. It is also 
observed in other electroweak processes.

\begin{figure}[t]
	\includegraphics[width=\smallfigwidth]{tex/signal/matching}
	\caption{Jet multiplicity produced by \meps{\powhegbox}{\pythia{8}} with a selection 
	of shower tunes. The green circles is the tune used in the analysis. The red squares 
	change the parton shower PDFs from CT10 to CTEQ6L1. The blue triangles additionally 
	change the parton shower $\alphaS\parenths{\mZ}$ from 0.137 to 0.118 (in agreement 
	with \powhegbox). \meps{\powhegbox}{\pythia{6}} is shown in black for reference.}
	\label{fig:signal:matching}
\end{figure}

The hadronisation and UE models of standalone \pythia{8} have been tuned to ATLAS 
UE data with a variety of PDF sets (known as AU2 tunes) \cite{ATLAS:tune:2012}.
However, the parton shower was not tuned since the default settings successfully described 
experimental data. 

When modelling ggF with \powhegbox, the AU2-CT10 tune was used in order to match the 
PDFs used in the matrix element calculation. Technically speaking, a dedicated 
\meps{\powhegbox}{\pythia{8}} tune should have been used, but this was unavailable. 
Unfortunately, a couple of issues had a negative impact on the NLO-PS matching. First, 
the parton shower evolves \alphaS at LO while NLO PDFs were used in the shower. 
Second, there was a mismatch between the \alphaS used in \powhegbox, 
$\alphaS\parenths{\mZ} = 0.118$, and the default value in the parton shower, 
$\alphaS\parenths{\mZ} = 0.137$. The effect of these issues is shown in 
\Figure~\ref{fig:signal:matching}.

Identification of this poor matching has led to improvements in the latest round of MC 
tuning, where dedicated \meps{\powhegbox}{\pythia{8}} tunes are fit using an adjusted 
parton shower \cite{ATLAS:tune:2013}.



\subsection{Fiducial region acceptances}
\label{sec:ggF:fiducial}

In order to measure the total ggF cross section, it is necessary to estimate the signal 
acceptance of the object and event selection. The extrapolation from the measured phase 
space to the inclusive phase space introduces theoretical uncertainties (see 
\Section~\ref{sec:ggF:acc}). It is helpful to separate these theoretical uncertainties from 
the others by measuring an intermediate cross section in a \textit{fiducial region} of 
phase space, chosen to be similar to that used in the detector-level selection in order to 
minimise the extrapolation.

The measured fiducial cross section is extracted by
\begin{equation}
	\sigma_{\text{ggF}}^{\text{fid}} = \frac{N_{\text{obs}} - N_{\text{bkg}}}{C_{\text{ggF}} \cdot L}
	\label{eq:ggF:fid_xs}
\end{equation}
where $N_{\text{obs}}$ is the observed number of events, $N_{\text{bkg}}$ is the expected 
number of background events, $L$ is the luminosity, and $C_{\text{ggF}}$ is the ratio of 
the expected number of ggF events passing the detector-level selection to those passing 
the fiducial selection. $C_{\text{ggF}}$ accounts for detector effects such as lepton 
trigger and reconstruction efficiencies and object mismeasurement due to the finite 
resolution of the detector.

On the other hand, the measured total cross section is extracted by
\begin{equation}
	\sigma_{\text{ggF}} = \frac{N_{\text{obs}} - N_{\text{bkg}}}{C_{\text{ggF}} \cdot A_{\text{ggF}} \cdot \text{BR} \cdot L}
	\label{eq:ww:total_xs}
\end{equation}
where $A_{\text{ggF}}$ is the ratio of the expected number of ggF events passing the 
fiducial selection to the total expected number of ggF events. $A_{\text{ggF}}$ accounts 
for the acceptance of the event selection criteria. BR incorporates the branching 
ratios of the Higgs and \PW bosons for the channel in question, and includes contributions 
from leptonic \Ptau decays.

Fiducial cross sections are extracted from the combined \emch{}+\mech channel only, since 
these are the most sensitive channels. Also, the split signal regions (by \ptsubleadlep and 
\mll) are combined in the fiducial cross section extraction. Apart from these two points, 
the event selection criteria of the fiducial regions are identical to those described in 
\Section~\ref{sec:ggF:acc}. The signal acceptances $C_{\text{ggF}}$, $A_{\text{ggF}}$ and 
$C_{\text{ggF}} \cdot A_{\text{ggF}}$ are displayed in \Table~\ref{tab:ggF:cggF_aggF}, 
together with their respective uncertainties.\todo{Add \twojet bin?}

\begin{table}[t]
	\begin{tabular}{l@{}c@{\hskip 0.2in}c@{\hskip 0.2in}c}
		\toprule
		& 0-jet & 1-jet & \twojet \\
		\midrule
		$C_{\text{ggF}}$ ($\times 100$) & $46.5\pm2.6$ & $45.6\pm1.8$ &  \\
		\quad Trigger efficiency              & 1.2\% & 1.0\% & \\
		\quad Lepton efficiency               & 3.1\% & 2.7\% & \\
		\quad Lepton \pt scale and resolution & 1.3\% & 1.3\% & \\
		\quad Jet energy scale and resolution & 4.5\% & 1.6\% & \\
		\quad Jet \Pbottom-tagging efficiency & 0.0\% & 1.8\% & \\
		\quad \met modelling                  & 0.2\% & 0.8\% & \\
		\cmidrule(lr){1-4}
		$A_{\text{ggF}}$ ($\times 100$) & $15.0\pm0.7$ & $7.1\pm0.4$ & \\
		\quad \mur and \muf scales & 1.1\% & 1.4\% & \\
		\quad PDFs                 & 3.7\% & 3.4\% & \\
		\quad PS/UE                & 2.2\% & 3.8\% & \\
		\quad NLO-PS               & 1.8\% & 1.1\% & \\
		\cmidrule(lr){1-4}
		$C_{\text{ggF}} \cdot A_{\text{ggF}}$ ($\times 100$) & $7.0\pm0.5$ & $3.2\pm0.2$ & \\
		\bottomrule
	\end{tabular}
	\caption{The signal acceptances $C_{\text{ggF}}$, $A_{\text{ggF}}$ and $C_{\text{ggF}} 
	\cdot A_{\text{ggF}}$. A breakdown of the relative uncertainties from different sources 
	is also shown.}
	\label{tab:ggF:cggF_aggF}
\end{table}

