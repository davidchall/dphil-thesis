%!TEX root = ../../thesis.tex

The \ac{SM} of particle physics is the theory of sub-atomic particles.
It is the culmination of many decades of progress in both experimental and theoretical
quantum physics during the 20th century. It has enjoyed unparalleled success in describing 
a wide range of phenomena, which have been experimentally verified to an extraordinary 
degree of precision.

A crucial aspect of the \ac{SM} is the generation of particle masses, which are forbidden 
by symmetry arguments though remain an experimental fact. These are generated by 
the Higgs mechanism of electroweak symmetry breaking, which also predicts the existence of 
a massive scalar particle, the Higgs boson, whose mass is a free parameter of the theory. 
As the only undiscovered particle of the \ac{SM}, the discovery of the the Higgs boson was 
a primary goal of the LHC physics program, which began in 2010.

A brief introduction to the \ac{SM} is given in \Section~\ref{sec:sm}, outlining the 
particle content and interactions of the theory. In \Section~\ref{sec:ewsb}, electroweak 
symmetry breaking is described in detail. Then, some properties of the Higgs boson are 
described in \Section~\ref{sec:properties}, and the constraints upon its mass prior to the 
\acs{LHC} are detailed in \Section~\ref{sec:prior_constraints}.
