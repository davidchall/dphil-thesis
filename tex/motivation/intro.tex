%!TEX root = ../../thesis.tex

The \ac{SM} of particle physics is the theory of the interactions of sub-atomic particles.
It is the culmination of many decades of progress in both experimental and theoretical
quantum physics during the 20th century. It has enjoyed unparalleled success in describing 
a wide range of phenomena, which have been experimentally verified to an extraordinary 
degree of precision.

A crucial aspect of the \ac{SM} is the generation of particle masses, which are forbidden
by the electroweak interaction. Instead, masses are generated by the Higgs mechanism of
electroweak symmetry breaking. This also predicts the existence of a massive scalar
particle, the Higgs boson, whose mass is a free parameter of the theory. At the start of 
the \acs{LHC} run in 2009, the Higgs boson was the only undiscovered particle of the 
\ac{SM}; thus its discovery was a primary goal of the \acs{LHC} physics program.

A brief introduction to the \ac{SM} is given in \Section~\ref{sec:sm}, outlining the 
particle content and interactions of the theory. In \Section~\ref{sec:ewsb}, electroweak 
symmetry breaking is described in detail. Then, some properties of the Higgs boson are 
described in \Section~\ref{sec:properties}, and the constraints upon its mass prior to the 
\acs{LHC} are detailed in \Section~\ref{sec:prior_constraints}.
