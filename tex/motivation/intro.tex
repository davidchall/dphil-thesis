%!TEX root = ../../thesis.tex

The Standard Model (SM) of particle physics describes the behaviour of sub-atomic particles. 
Since its formulation in the 1970s, it has experienced unparalleled success in modelling a 
wide range of phenomena that have been experimentally verified to an extraordinary 
degree of precision; no experimental result within the remit of the Standard Model is 
currently considered to significantly contradict its validity.\footnote{
	Observation of neutrino oscillations required neutrino masses to be manually added to 
	the Standard Model. It is widely believed that their relatively small masses will be 
	explained by new physics.
}
However, there are a number of physical phenomena that the Standard Model is unable to 
describe: gravitational attraction between massive objects, the observed asymmetry between 
matter and antimatter in the Universe, and astronomical evidence for dark matter and the 
cosmological constant.

A crucial aspect of the SM is how non-zero masses are imparted to fundamental particles. 
These are forbidden by underlying symmetries of the theory, though remain an experimental 
fact; for example, atoms could not form if the electron did not possess mass. This is 
achieved via interactions with a ubiquitous Higgs field, excitations of which correspond to 
Higgs bosons. As the only undiscovered particle of the SM, the discovery of the Higgs boson 
is of utmost importance to particle physics: it would complete our knowledge of the SM, and 
in particular confirm the mechanism of mass generation. As such, it was a primary goal of the 
LHC physics program, which began in 2010.

A brief introduction to the SM is given in \Section~\ref{sec:sm}, outlining the 
particle content and interactions of the theory. In \Section~\ref{sec:ewsb}, electroweak 
symmetry breaking is described in detail. Then, some properties of the Higgs boson are 
described in \Section~\ref{sec:properties}, and the constraints upon its mass prior to the 
LHC are detailed in \Section~\ref{sec:prior_constraints}.
