%!TEX root = ../../thesis.tex

The \ac{SM} is a gauge quantum field theory describing the kinematics and interactions of 
sub-atomic particles \cite{Aitchison,Peskin}. The dynamics of such a theory are determined 
by the symmetries respected by its Lagrangian. The \ac{SM} is invariant under local 
transformations of the \SMgroup gauge group, resulting in the strong, weak and 
electromagnetic forces of nature. Additionally, invariance under global transformations of 
the Poincaré group ensures the theory is identical in all inertial frames of reference, as 
asserted by special relativity.

Each constituent gauge theory of the \ac{SM} describes the dynamics of a force of nature, 
which is mediated by a number of gauge bosons and couples to a conserved current, in 
accordance with Noether's theorem. \ac{QCD} of \SUgroup{3} describes the strong 
interaction, is mediated by eight gluons and couples to colour charge. \ac{QED} of 
\Ugroup{1} describes the electromagnetic interaction, is mediated by the 
photon and couples to electric charge. The weak interaction is mediated by the massive 
\PWpm and \PZ bosons and is best understood within the context of \ac{EW} theory,
a unification of the electromagnetic and weak interactions. A theory of
gravity is not included in the \ac{SM}. Significantly, the gauge groups of 
the strong and weak interactions are non-abelian. Physically, this means that the
gauge bosons are themselves charged and therefore experience self-interactions.

\begin{figure}
	\includegraphics[width=\largefigwidth]{tex/motivation/sm_particles}
	\caption{The particle content of the \ac{SM}, with masses from \cite{PDG}. Constraints
	upon the mass of the Higgs boson are described in \Section~\ref{sec:prior_constraints}.}
	\label{fig:sm_particles}
\end{figure}

The elementary particles of the \ac{SM} are summarised in \Figure~\ref{fig:sm_particles}.
They are categorised into bosons (integer spin) and fermions (half-integer spin).
In addition to the gauge bosons introduced, the Higgs boson is a by-product
of electroweak symmetry breaking (described in \Section~\ref{sec:ewsb}) and couples to 
mass. The twelve flavours of fermions are categorised according to the interactions they 
experience, or equivalently the charges they posses: quarks (strong, electromagnetic, 
weak), charged leptons (electromagnetic, weak) and neutrinos (weak). The fermions are also 
arranged in three generations of increasing mass. Massive particles can decay into less 
massive particles, while observing the conservation laws of the \ac{SM}. Where possible, 
particles have an associated antiparticle with identical mass but inverted internal 
quantum numbers.
