%!TEX root = ../../thesis.tex

This section describes the \WW cross section measurement using the 
\unit{$\sqrt{s} = 7$}{\TeV} dataset, which is published in \Reference~\cite{WW-7TeV}.

\WW has the same basic experimental signature as \HWW, and therefore shares many of its 
backgrounds. However, there are two key differences between the analyses. 

\begin{enumerate}
	\item The analysis described in \Chapter~\ref{chap:selection} is optimised for low 
	mass (\unit{$\mH \approx 125$}{\GeV}) resonant \WW production. It therefore requires 
	at least one \PW boson to be off-shell, and consequently employs low lepton 
	thresholds (\unit{$\pt > 10$}{\GeV}). Conversely, the search for non-resonant \WW 
	production is optimised for on-shell \PW bosons, and so raises the lepton thresholds 
	to \unit{$\pt > 20$}{\GeV}. This reduces fake leptons, and obviates the need to split 
	the different flavour channel into \emch and \mech parts.

	\item The total \WW cross section at \unit{$\sqrt{s} = 7$}{\TeV} is 
	\unit{$44.7^{+2.1}_{-1.9}$}{\pico\barn}, which is much larger than than of \ggHWW 
	(\unit{$3.3 \pm 0.4$}{\pico\barn} for \unit{$\mH = 125$}{\GeV}). This enables use of 
	tighter criteria to reduce backgrounds, whilst retaining a large number of signal 
	events. For example, only the 0-jet bin is used.
\end{enumerate}



\subsection{Reconstruction of physics objects}

Beam conditions in 2011 were quite different to 2012; in particular, the pile-up 
environment was much less difficult (see \Figure~\ref{fig:dataset:pileup}). Consequently, 
the reconstruction of physics objects is slightly different to that described in 
\Section~\ref{sec:objects}.

\begin{description}
\item[Electrons] \hfill \\
	The Gaussian Sum Filter was not implemented in the 2011 reconstruction, and so the 
	efficiency is lower (see \Figure~\ref{fig:objects:el_recoeff}). To reduce fakes, the 
	cut-based \textit{tight} identification criteria are used. The reduced pile-up allows
	tighter calorimeter isolation to be applied, $\etcone{0.3}/\et < 0.14$, and this in 
	turn allows for looser tracker isolation $\ptcone{0.3}/\et < 0.13$. Finally, the 
	association with the primary vertex is relaxed, with the transverse impact parameter 
	$d_0$ required to be within ten standard deviations of zero. The threshold is raised 
	to \unit{$\pt > 20$}{\GeV}.

\item[Muons] \hfill \\
	Differences to muon reconstruction are minimal. Slightly tighter quality criteria are 
	applied to the ID tracks. The isolation criteria are $\etcone{0.3}/\pt < 0.14$ and 
	$\ptcone{0.3}/\pt < 0.15$. The threshold is raised to \unit{$\pt > 20$}{\GeV}.

\item[Jets] \hfill \\
	A lower pile-up noise threshold is used in topo-clustering, corresponding to 
	$\mu = 8$. Local cluster weighting (LCW) is not performed on topo-clusters, and so 
	jets are corrected directly from the EM scale to the Jet Energy Scale (JES). In 2011, 
	the pile-up subtraction step of the calibration is less sophisticated, and is 
	averaged over \npv and $\mu$ rather than as an event-by-event correction 
	\cite{Jets:PileupCorrection:2011}. The jet vertex fraction criterion is also removed. 
	Finally, a unified threshold of \unit{$\pt > 25$}{\GeV} is used over the entire 
	range $\mods{\eta} < 4.5$.

\item[Missing transverse momentum] \hfill \\
	Calorimeter-based \met is used exclusively throughout the analysis.

\end{description}



\subsection{Event selection criteria}

As mentioned above, the event selection of the \WW measurement can be tighter than that 
of the \HWW search, and does not require such stringent optimisation.

\begin{description}
\item[Data quality] \hfill \\
	The 2011 \pp dataset (see \Section~\ref{sec:dataset:dataset}) is subject to data 
	quality criteria as in 2012, though some criteria are specific to the data-taking 
	conditions of 2011. The selected dataset corresponds to an integrated luminosity of 
	\unit{4.6}{\invfb}.

\item[Trigger] \hfill \\
	The lowest unprescaled single lepton triggers were used to support 
	\unit{$\ptleadlep > 25$}{\GeV} in the offline analysis, whilst operating on the 
	plateau. These changed throughout the year as data-taking conditions changed, and 
	are displayed in \Table~\ref{tab:ww:triggers}.

	\begin{table}
		\begin{tabular}{ll@{\hskip 0.3in}r@{\;{--}\;}l}
			\toprule
			\multirow{3}{*}{\Pe}  & \verb|EF_e20_medium|     & 14th Apr & 4th Aug \\
			                      & \verb|EF_e22_medium|     & 4th Aug & 22nd Aug \\
			                      & \verb|EF_e22vh_medium1|  & 7th Sep & 30th Oct \\
			\midrule
			\multirow{2}{*}{\Pmu} & \verb|EF_mu18_MG|        & 14th Apr & 29th Jul \\
			                      & \verb|EF_mu18_MG_medium| & 30th Jul & 30th Oct \\
			\bottomrule
		\end{tabular}
		\caption{Single lepton triggers employed in the 2011 \WW cross section 
		measurement. Trigger names are explained in \Table~\ref{tab:sel:triggers}.}
		\label{tab:ww:triggers}
	\end{table}

\item[Event selection] \hfill \\
	The event selection is shown in \Table~\ref{tab:ww:sel} and is similar to the 0-jet 
	\HWW selection in \Table~\ref{tab:event_selection}, although the topological cuts of 
	the Higgs boson decay are obviously not applied. The major differences are the raised 
	lepton thresholds, the raised \met cuts, and the lack of \dphillmet and \frecoil cuts.

	\begin{table}[b]
		\begin{tabularx}{0.5\textwidth}{YY}
			\toprule
			\emch & \eech/\mmch \\
			\midrule
			\multicolumn{2}{c}{$\ptleadlep > 25$ and $\ptsubleadlep > 20$} \\
			$\mll > 15$    & $\mll > 10$ \\
			--             & $\mods{\mll - \mZ} > 15$ \\
			$\metrel > 25$ & $\metrel > 45$ \\
			\multicolumn{2}{c}{$\njets = 0$} \\
			\multicolumn{2}{c}{$\ptll > 30$} \\
			\bottomrule
		\end{tabularx}
		\caption{Summary of \WW event selection. Cuts are given in \GeV.}
		\label{tab:ww:sel}
	\end{table}
\end{description}






