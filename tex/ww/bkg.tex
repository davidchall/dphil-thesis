%!TEX root = ../../thesis.tex

Non-resonant \WW production is an irreducible background to the \HWW search, and is the 
dominant background contribution following the event selection described in 
\Chapter~\ref{chap:selection}. Thus, it is critical that it is accurately estimated. 

To this end, a data-driven technique is employed whereby MC is used to extrapolate from a 
control region (CR) to the signal region (SR):
\begin{equation}
	N_{\WW}^{\text{pred,SR}} &= \alpha_{\WW} \parenths{N_{\WW}^{\text{data,CR}} - N_{\text{non-\WW}}^{\text{pred,CR}} - N_{\text{sig}}^{\text{MC,CR}}} \\
	\alpha_{\WW} &= N_{\WW}^{\text{MC,SR}} / N_{\WW}^{\text{MC,CR}}
\end{equation}
where $N_{\text{non-\WW}}^{\text{pred,CR}}$ is determined by dedicated methods (see 
\Chapter~\ref{chap:backgrounds}). The CR is optimised to reduce 
$N_{\text{non-\WW}}^{\text{pred,CR}}$ and $N_{\text{sig}}^{\text{MC,CR}}$ to avoid bias, 
whilst maintaining a large $N_{\WW}^{\text{data,CR}}$ to minimise the statistical 
uncertainty in $N_{\WW}^{\text{pred,SR}}$. The signal contamination 
$N_{\text{sig}}^{\text{MC,CR}}$ is actually determined in the fitting procedure since the 
signal cross section is a fit parameter. The MC-based extrapolation $\alpha_{\WW}$ 
introduces theoretical uncertainties to $N_{\WW}^{\text{pred,SR}}$.

The \WW process is modelled at NLO by \meps{\powhegbox}{\pythia{6}}. It is necessary to 
use \powhegbox since \mcatnlo does not feature ``singly resonant diagrams'', where the 
mass of the dilepton + dineutrino system is that of a single \PW boson. Since the \HWW 
search is sensitive to off-shell \PW bosons, it is important to include these diagrams. 
The \pythia{6} event generator is used as it gives a better description of the 
experimental data in many exclusive observables, when compared to \pythia{8}. This is likely to be related to the \meps{\powhegbox}{\pythia{8}} matching issues mentioned in 
\Section~\ref{sec:ggF:meps_matching}. Also, the ATLAS underlying event AUET2B tune 
\cite{ATLAS:tune:2011} was found to be overtuned to dijet data and consequently its 
description of electroweak processes suffered. For this reason, the updated Perugia 2011C 
\pythia{6} tune \cite{PerugiaTunes} was used. NNLO \ggWW diagrams are modelled by 
\meps{\ggtoww}{\fherwig} \cite{gg2ww}.

As explained in \Section~\ref{sec:selection:higgs_decay}, the resonant \HWW and 
non-resonant \WW processes are distinguished by the scalar nature of the Higgs boson, 
which affects the decay topology of the \PW bosons. This causes \HWW events to have a 
small dilepton opening angle, and consequently low \mll and \dphill. It is therefore 
intuitive to define a \WW CR at high \mll, with a relaxed \dphill requirement.

Since jet binning can introduce large uncertainties, a \WW CR is defined for each of the 
0-jet and 1-jet bins separately. Unfortunately, it is not possible to define a \WW CR in 
the \twojet bin with sufficient purity owing to the large top background. Thus the \WW 
background estimation in the \twojet bin is MC-based (see \Section~\ref{sec:ww_bkg:2j}).

The event selection of the \WW CRs are outlined in \Table~\ref{tab:ww_cr}. Since the 
\eech/\mmch channels are contaminated by \DYll, CRs are defined in the \emch/\mech 
channels only. The \eech/\mmch 0-jet \WW prediction is extrapolated from the \emch/\mech 
0-jet \WW CR, and similarly for the 1-jet bin. The selection criteria mostly follow those 
of the signal regions (see \Table~\ref{tab:event_selection}) with an \mll lower bound and 
a loosened \dphill cut. However, the \ptsubleadlep threshold is raised from \unit{10}{\GeV}
to \unit{15}{\GeV} in order to reduce contamination from the \Wjets background. An 
additional validation region (VR) is defined in the 0-jet bin with 
\unit{$\mll > 110$}{\GeV}, in order to validate the extrapolation from the CR.

\begin{table}[t]
	\begin{tabularx}{0.55\textwidth}{YY}
		\toprule
		\multicolumn{2}{c}{\emch/\mech} \\
		\midrule
		\multicolumn{2}{c}{$\ptleadlep > 22$ and $\ptsubleadlep > 15$} \\
		\multicolumn{2}{c}{$\corrtrackmet > 20$} \\
		\cmidrule(lr){1-2}
		0-jet bin & 1-jet bin \\
		\cmidrule(lr){1-2}
		$\ptll > 30$ & $\mods{\mtautau - \mZ} > 25$ \\
		$\dphillmet > \pi/2$ & $\maxmtw > 50$ \\
		-- & $\nbjets = 0$ \\
		$55 < \mll < 110$ & $\mll > 80$ \\
		$\dphill < 2.6$ & -- \\
		\bottomrule
	\end{tabularx}
	\caption{Event selection criteria of the \WW control regions (unavailable in the 
	\twojet bin). Cuts on energy, momentum and mass are given in \GeV, and angular cuts 
	are given in radians. The relevant variables are described in 
	\Chapter~\ref{chap:selection}.}
	\label{tab:ww_cr}
\end{table}

A complementary view of the \WW background estimation is of a normalisation factor 
derived in the CR. This can improve the \WW estimation is regions of phase space other 
than the signal region, which would otherwise be MC-based. However, theoretical 
uncertainties in the associated extrapolation are not calculated, so this is more limited 
than the prediction in the SR. The normalisation factors are measured to be 
\stat{1.22}{0.03} in the 0-jet bin and \stat{1.06}{0.05} in the 1-jet bin\todo{update}. 
Some distributions within the CRs are displayed in \Figure~\ref{fig:ww_bkg:cr_plots}, to 
show the excellent description of experimental data following application of the
normalisation factors.

\begin{figure}
	\includegraphics[width=0.495\textwidth]{tex/ww/emme_CutWWControl_0jet_DPhill_mh125_lin}
	\hfill
	\includegraphics[width=0.495\textwidth]{tex/ww/emme_CutWWControl_0jet_MT_TrackHWW_Clj_mh125_lin}
	\\
	\includegraphics[width=0.495\textwidth]{tex/ww/emme_CutWWControl_1jet_DPhill_mh125_lin}
	\hfill
	\includegraphics[width=0.495\textwidth]{tex/ww/emme_CutWWControl_1jet_MT_TrackHWW_Clj_mh125_lin}
	\caption{The \dphill (left) and \mt (right) distributions in the 0-jet (top) and 
	1-jet (bottom) \WW control regions. The \WW normalisation factor has been applied in
	these plots.}
	\label{fig:ww_bkg:cr_plots}
\end{figure}


\subsection{Theoretical uncertainties in $\alpha_{\WW}$}
\label{sec:ww_bkg:alpha}

\begin{table}
	\centering
	\begin{tabular}{ccc|ccccc}
		\toprule
		& \mll & \ptsubleadlep & \multirow{2}{*}{Scale} & \multicolumn{2}{c}{PDF} & \multirow{2}{*}{PS/UE} & \multirow{2}{*}{NLO-PS} \\
		& (\GeV) & (\GeV) & & collab. & 68\% CL & & \\
		\midrule
		\multicolumn{8}{c}{\eech/\mmch channels} \\
		\midrule
		0-jet & 12--55 & $>10$ & 0.8\% & 0.5\% & 1.0\% & $-1.2\%$ & $+2.4\%$ \\
		1-jet & 12--55 & $>10$ & 0.8\% & 0.5\% & 0.7\% & $-2.3\%$ & $+3.8\%$ \\
		\midrule
		\multicolumn{8}{c}{\emch/\mech channels} \\
		\midrule
		\multirow{6}{*}{0-jet}
		& \multirow{3}{*}{10--30}
	    &  10--15 & 0.7\% & 0.9\% & 0.2\% & $+2.2\%$ & $+0.4\%$ \\
		&& 15--20 & 1.2\% & 0.8\% & 0.2\% & $+1.7\%$ & $+0.9\%$ \\
		&&  $>20$ & 0.7\% & 0.5\% & 0.3\% & $-1.9\%$ & $+3.1\%$ \\
		\cmidrule(lr){2-8}
		& \multirow{3}{*}{30--55}
		&  10--15 & 0.7\% & 0.8\% & 0.1\% & $+1.5\%$ & $+0.5\%$ \\
		&& 15--20 & 0.8\% & 0.7\% & 0.2\% & $+1.0\%$ & $+1.0\%$ \\
		&&  $>20$ & 0.8\% & 0.4\% & 0.5\% & $-2.4\%$ & $+3.9\%$ \\
		\cmidrule(lr){1-8}
		\multirow{6}{*}{1-jet}
		& \multirow{3}{*}{10--30}
	    &  10--15 & 3.1\% & 0.5\% & 0.1\% & $-2.4\%$ & $-3.4\%$ \\
		&& 15--20 & 1.6\% & 0.5\% & 0.1\% & $-3.0\%$ & $+0.7\%$ \\
		&&  $>20$ & 1.0\% & 0.6\% & 0.2\% & $-3.6\%$ & $+5.3\%$ \\
		\cmidrule(lr){2-8}
		& \multirow{3}{*}{30--55}
		&  10--15 & 3.2\% & 0.5\% & 0.1\% & $-2.0\%$ & $+1.9\%$ \\
		&& 15--20 & 1.5\% & 0.4\% & 0.1\% & $-3.0\%$ & $+2.4\%$ \\
		&&  $>20$ & 1.3\% & 0.5\% & 0.4\% & $-3.1\%$ & $+5.6\%$ \\
		\midrule
		\multicolumn{8}{c}{Validation region (\emch/\mech channels)} \\
		\midrule
		0-jet & $>110$ & $>15$ & 0.6\% & 0.6\% & 2.0\% & $+4.3\%$ & $-5.1\%$ \\
		\bottomrule
	\end{tabular}
	\caption{Theoretical uncertainties in the \WW extrapolation parameter $\alpha_{\WW}$ 
	for each signal region used in the fitting procedure, and also for the validation 
	region.}
	\label{tab:ww_bkg:alpha_unc}
\end{table}


\subsection{\mt shape modelling}
\label{sec:ww_bkg:mt}


\subsection{\WW background in the \twojet bin}
\label{sec:ww_bkg:2j}

