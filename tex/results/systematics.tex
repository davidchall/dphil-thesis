%!TEX root = ../../thesis.tex

In addition to statistical uncertainties in the observed number of events and the expected 
number of events (due to finite MC sample sizes), multiple sources of systematic uncertainty 
should be considered. Many of these were introduced throughout 
Chapters~\ref{chap:selection}--\ref{chap:backgrounds}, but a short summary is presented here.



\subsection{Experimental uncertainties}
\label{sec:syst:exp}

Experimental uncertainties arise due to a mismodelling of the detector performance, 
particularly in the reconstruction efficiency and energy calibration of physics objects. 
These introduce uncertainties either directly into the expected acceptance (as with signal) 
or via the extrapolation from a control region (as with many backgrounds).

\begin{description}
\item[Trigger efficiency] \hfill \\
	The lepton trigger efficiencies and their uncertainties are measured as a function of 
	\pt and $\eta$ using tag-and-probe of \HepProcess{\PZ \HepTo \Plepton\Plepton} events 
	(see \Section~\ref{sec:selection:trigger}). This includes the efficiency of matching the 
	trigger to the lepton object.
	
\item[Lepton reconstruction efficiency] \hfill \\
	The lepton selection efficiencies and their uncertainties are measured as a function of 
	\pt and $\eta$ using tag-and-probe of \HepProcess{\PZ \HepTo \Plepton\Plepton} events 
	(see Sections~\ref{sec:objects:electrons} and \ref{sec:objects:muons}). This is 
	performed separately for each step in the lepton selection, \ie reconstruction, 
	identification, isolation and primary vertex association. The efficiency uncertainties 
	are $<\!0.5\%$ for muons and $<\!3\%$ for electrons.
	
\item[Lepton energy scale and resolution] \hfill \\
	Lepton energy scales and resolutions are calibrated \textit{in situ} as a function of 
	\pt and $\eta$ using \HepProcess{\PJpsi \HepTo \Plepton\Plepton} and \HepProcess{\PZ 
	\HepTo \Plepton\Plepton} resonance events, and the associated uncertainties are also 
	derived during this calibration. Scale uncertainties are $<\!0.5\%$ and their impact 
	is assessed by varying lepton energies by $\pm1\sigma$. Resolution uncertainties are 
	$<\!1\%$ and their impact is assessed by varying the smearing of lepton energies by 
	$\pm1\sigma$.

\item[Jet reconstruction efficiency] \hfill \\
	The jet selection criteria feature a cut upon the jet vertex fraction (JVF) in order to 
	reduce pile-up jets (see \Section~\ref{sec:objects:jets}). Unlike the trigger, lepton 
	reconstruction and \Pbottom-tagging efficiencies, the JVF efficiency is not subject to 
	a data-driven scale factor and so MC mismodelling introduces a systematic uncertainty to 
	the \njets distribution. This is only an issue in signal events, since the data-driven 
	background estimations are jet-binned. The mismodelling is evaluated as 1.2\% in a \DYll 
	control region, producing negligible jet bin migrations.

\item[Jet energy scale and resolution] \hfill \\
	The jet energy scale (JES) is calibrated as a function of \pt and $\eta$, as described 
	in \Section~\ref{sec:objects:jets}. The \textit{in situ} calibration method of balancing 
	jets against well-measured reference objects introduces uncertainties to the JES. Also, 
	uncertainties in the pile-up environment introduce uncertainties to the JES through the 
	pile-up subtraction step of the calibration. Finally, the JES is affected by 
	uncertainties in the jet flavour composition and the detector response to each flavour.
	The uncertainty components are treated separately (the total uncertainty is $<\!7\%$) 
	and their impact is assessed by varying jet energies by $\pm1\sigma$.

	The jet energy resolution (JER) and its uncertainty are measured \textit{in situ} as a 
	function of \pt and $\eta$ using dijet events \cite{Jets:JER:2011}. The bisector method 
	projects the dijet \pt imbalance in the direction bisecting the two jets and in the 
	orthogonal direction. At hadron-level the two components are expected to have the same 
	variance, but at detector-level the variance of the latter is increased by the JER.
	Thus, measurements of the two components allow the JER to be evaluated. The uncertainty 
	is $<\!10\%$ and its impact is assessed by increasing the smearing of jet energies 
	by $+1\sigma$.

\item[Jet \Pbottom-tagging efficiency] \hfill \\
	The \Pbottom-tagging efficiency for \Pbottom-jets is calibrated as a function of jet 
	\pt, using dileptonic \ttbar decays in a combinational likelihood method \cite{Btag:llh}.
	At \unit{$\pt \approx 20$}{\GeV} the measurement is limited by JES uncertainties, but at 
	\unit{$\pt > 40$}{\GeV} it is limited by modelling uncertainties in the jet flavour 
	composition. Uncertainties in the \Pbottom-tagging efficiency of \Pcharm-jets and light 
	jets are also considered, though have little effect.

\item[\met modelling] \hfill \\
	As described in \Section~\ref{sec:objects:met}, \calomet, \trackmet and \corrtrackmet 
	are defined using the calibrated electron, muon and jet objects, and are therefore 
	correlated to the electron, muon and jet energy scales whose uncertainties are outlined 
	above. In the \calomet calculation, uncertainties in the scale of the unassociated soft 
	calorimeter deposits are also considered. In the \trackmet and \corrtrackmet 
	calculations, uncertainties in the nett $\bvec{p}_{\text{T}}$ of unassociated tracks 
	(\eg due to neutral particles) are also considered.

\item[Lepton fake factors] \hfill \\
	The lepton fake factors $f_{\Plepton}$ are important to the \Wjets and dijet background 
	estimations, and are detailed in \Section~\ref{sec:wjets}. $f_{\Plepton}^{\text{\Wjets}}$
	is measured in \Zjets events, and is dominated by statistical uncertainties, 
	uncertainties in the electroweak background subtraction and theoretical uncertainties in 
	MC-based corrections. However, a component of the uncertainties cancels in the same-sign 
	control region method (see \Section~\ref{sec:diboson:sscr}). 
	$f_{\Plepton}^{\text{dijet}}$ is measured in dijet events and is dominated by 
	theoretical uncertainties in MC-based corrections.

\item[Pile-up] \hfill \\
	Uncertainties in the pile-up environment affect the jet calibration, as outlined above. 
	Pile-up can also introduce additional hard jets that lead to migrations between jet bins.
	Uncertainties in pile-up lead to migration uncertainties of 0.5\% and 1.0\% in the 0-jet 
	and 1-jet bins respectively, and are neglected as they are much smaller than the ggF 
	jet binning uncertainties.

	The effect of out-of-time pile-up upon detector electronics might also be mismodelled. 
	This is found to be negligible.

\item[Luminosity] \hfill \\
	The luminosity measurement is calibrated during van der Meer scans (see 
	\Section~\ref{sec:dataset:lumi}), with a relative uncertainty of 2.8\% for the 
	\unit{$\sqrt{s} = 8$}{\TeV} dataset.

\end{description}



\subsection{Theoretical uncertainties}
\label{sec:syst:theor}

Many theoretical uncertainties are also considered (and indeed are included within the 
experimental uncertainties above). Theoretical uncertainties in the expected signal are 
described in \Chapter~\ref{chap:signal}. Theoretical uncertainties in the expected 
background appear when some aspect of a data-driven technique relies upon MC modelling. 
Some of the uncertainties considered are presented in Chapters~\ref{chap:ww} and 
\ref{chap:backgrounds}.

The sources of uncertainty usually considered are: higher order corrections in the 
perturbative series, incoming parton distribution functions, and aspects of the MC modelling 
(\eg hadronisation and underlying event models).

