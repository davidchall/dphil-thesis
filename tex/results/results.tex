%!TEX root = ../../thesis.tex

The entire LHC Run~I dataset was analysed, corresponding to an integrated luminosity of 
\unit{4.5}{\invfb} at \unit{$\sqrt{s} = 7$}{\TeV} and \unit{20.3}{\invfb} at 
\unit{$\sqrt{s} = 8$}{\TeV}. The major differences of the \unit{7}{\TeV} analysis, with 
respect to \Chapter~\ref{chap:selection}, are the absence of the \twojet bin and the 
dilepton triggers, and the use of a dijet fake factor in the \Wjets background estimation. 
Additionally, the number of \mt bins used in the fit is reduced by a factor of two.
Some minor differences also exist in the object and event selection criteria due to the less 
harsh pile-up environment at \unit{7}{\TeV} (see \Section~\ref{sec:dataset:dataset}).



\subsection{Exclusion, discovery and measurement of \ggHWW}
\label{sec:results:ggF_limits}

The event selection criteria of the \ggHWW analysis are chosen such that the dominant 
production mode of the selected Higgs boson events is ggF. Nevertheless, other production 
modes can contribute to the signal acceptance. In the 0-jet and 1-jet bins this effect is 
small, but in the \twojet bin the VBF (\VH) process contributes 15\% (10\%) of signal events.
Such events are treated as signal and their expected yield is scaled by the same $\mu$ as 
ggF. This has a small effect upon the results, and splitting $\mu$ according to production 
mode shall be revisited in \Section~\ref{sec:results:combined_limits}.

The observed and expected (under a background-only hypothesis) 95\% CL upper limit on $\mu$ 
is shown as a function of \mH in \Figure~\ref{fig:ggf_results:CLs}. The mass range where 
this limit is below unity is considered excluded at the 95\% CL, when considering an SM 
Higgs boson ($\mu = 1$). In the absence of a Higgs boson, the expected excluded region is 
\unit{116}{\GeV} to \unit{200}{\GeV}.\footnote{
	This analysis is optimised to search for a low-mass Higgs boson and therefore the region 
	\unit{$\mH > 200$}{\GeV} is not considered. A dedicated high-mass search for \HWW is 
	described in \Reference~\cite{HWW-highmass}.
}
However, the observed excluded region is \unit{132}{\GeV} to \unit{200}{\GeV}.
The fact that the observed exclusion is weaker than expected is indicative of an excess of 
events consistent with a Higgs boson. Since the mass resolution of the \HWW analysis is 
poor, the impact upon the exclusion is broad in \mH.

To quantify the significance of the excess of events, \Figure~\ref{fig:ggf_results:p0} shows 
the observed $p_0$ as a function of \mH. The maximum observed significance is $4.8\sigma$, 
which occurs when testing \unit{$\mH = 130$}{\GeV}, though the poor mass resolution leads to 
a relatively broad $p_0$ curve. This is very strong evidence for the existence of the Higgs 
boson.

\begin{figure}[t]
	\begin{subfigure}[b]{0.495\textwidth}
		\centering
		\includegraphics[width=\textwidth,clip=true,trim=0.6cm 0.8cm 1.0cm 0.4cm]{custom_images/limits/cls_ggf_only}
		\caption{Exclusion}
		\label{fig:ggf_results:CLs}
	\end{subfigure}
	\hfill
	\begin{subfigure}[b]{0.495\textwidth}
		\centering
		\includegraphics[width=\textwidth,clip=true,trim=0.6cm 0.8cm 1.0cm 0.4cm]{custom_images/limits/p0_ggf_only}
		\caption{Discovery}
		\label{fig:ggf_results:p0}
	\end{subfigure}
	\\[12pt]
	\begin{subfigure}[b]{0.495\textwidth}
		\centering
		\includegraphics[width=\textwidth,clip=true,trim=0.6cm 0.8cm 1.0cm 0.4cm]{custom_images/limits/mu_ggf_only}
		\caption{Measurement of $\mu$}
		\label{fig:ggf_results:mu}
	\end{subfigure}
	\caption{(a) The observed (solid) 95\% CL upper limit on the signal strength $\mu$ as a 
	function of the mass under test \mH, and the expectation (dashed) under the 
	background-only hypothesis. (b) The observed (solid) $p_0$ as a function of \mH and the 
	expectation (dashed) under the signal-plus-background hypothesis with $\mu = 1$ and a 
	hypothesised mass equal to that under test.	(c) The best-fit signal strength $\hat{\mu}$ 
	(solid) as a function of \mH, with the 68\% CL interval shown (blue).}
\end{figure}

The best-fit signal strength $\hat{\mu}$ is shown as a function of the \mH under test in 
\Figure~\ref{fig:ggf_results:mu}. A range of Higgs boson masses are consistent with the 
assumption that the Higgs boson behaves as predicted by the SM (\ie $\mu = 1$). The data 
are also consistent with a lower mass Higgs boson with $\mu > 1$ or a higher mass Higgs 
boson with $\mu < 1$. 



\subsection{Combination with VBF analysis}
\label{sec:results:combined_limits}


By including \mH as a parameter of interest in the fit, it is possible 
to test which $\parenths{\mu, \mH}$ pair are most favoured by the data. This is presented in 
\Figure~\ref{fig:results:mu_mH}, together with the 68\% CL and 95\% CL contours. This 
clearly demonstrates that the analysis is more sensitive to $\mu$ than to \mH.

The updated results exhibit more than $5\sigma$ significance for the \ggHWWlvlv process, and 
consequently for the existence of the Higgs boson itself. LHC searches for 
\HepProcess{\PHiggs \HepTo \Pphoton\Pphoton} and \HepProcess{\PHiggs \HepTo \PZ\PZ} have 
found consistent evidence, which is summarised in \Section~\ref{sec:searches}. These two 
channels yield much better mass sensitivity, and observe \unit{$\mH \approx 125$}{\GeV}. 
For this reason, the above results are now reinterpreted in terms of a Higgs boson with 
\unit{$\mH = 125$}{\GeV}.

With \unit{$\mH = 125$}{\GeV}, the observed significance is $5.3\sigma$ and the expected 
significance is $5.0\sigma$. The measured signal strength is $\hat{\mu} = 1.14 \pm 0.28$. 
The significance and $\hat{\mu}$ observed in various signal regions are summarised in 
\Table~\ref{tab:results:sig_mu_breakdown}. This shows that the combined observation is 
mostly driven by the \emch and \mech channels of the 0-jet and 1-jet bins, in the 
\unit{$\sqrt{s} = 8$}{\TeV} dataset.

\begin{table}
	\begin{tabular}{l@{\hskip 0.4in}cc@{\hskip 0.4in}cc}
		\toprule
		& \multicolumn{2}{c@{\hskip 0.4in}}{\unit{$\sqrt{s} = 7$}{\TeV}} & \multicolumn{2}{c}{\unit{$\sqrt{s} = 8$}{\TeV}} \\
		& $Z_{\text{obs}}$ ($Z_{\text{exp}}$) & $\hat{\mu}_{\text{obs}}$ & $Z_{\text{obs}}$ ($Z_{\text{exp}}$) & $\hat{\mu}_{\text{obs}}$ \\
		\midrule
		Total                    & 1.8 (1.8) & $0.99^{+0.00}_{-0.00}$ & 4.9 (4.4) & $1.18^{+0.31}_{-0.27}$ \\
		\quad 0-jet              & 1.2 (1.5) & $0.75^{+0.00}_{-0.00}$ & 4.2 (3.6) & $1.26^{+0.42}_{-0.35}$ \\
		\quad\quad \emch{}+\mech & 0.9 (1.4) & $0.61^{+0.00}_{-0.00}$ & 4.4 (3.6) & $1.41^{+0.45}_{-0.38}$ \\
		\quad\quad \eech{}+\mmch & 1.0 (0.7) & $1.40^{+0.00}_{-0.00}$ & 0.2 (1.2) & $0.18^{+0.83}_{-0.77}$ \\
		\quad 1-jet              & 1.8 (1.1) & $1.70^{+0.00}_{-0.00}$ & 2.5 (2.5) & $1.02^{+0.52}_{-0.43}$ \\
		\quad\quad \emch{}+\mech & 2.0 (1.0) & $2.20^{+0.00}_{-0.00}$ & 2.8 (2.5) & $1.21^{+0.57}_{-0.46}$ \\
		\quad\quad \eech{}+\mmch & 0.1 (0.6) & $0.17^{+0.00}_{-0.00}$ & 0.4 (1.0) & $0.38^{+1.09}_{-1.04}$ \\
		\quad \twojet            & --        & --                     & 1.4 (1.2) & $1.22^{+0.96}_{-0.85}$ \\
		\bottomrule
	\end{tabular}
	\caption{The observed and expected significances in units of Gaussian standard 
	deviations, $Z$, and the measured signal strength for a Higgs boson with 
	\unit{$\mH = 125$}{\GeV}.}
	\label{tab:results:sig_mu_breakdown}
\end{table}

%Table of uncertainties on $\mu$



\subsection{Cross section measurements}
\label{sec:results:xs}

Following the method described in \Section~\ref{sec:ggF:acc}, the fiducial cross section is 
measured using the \unit{$\sqrt{s} = 8$}{\TeV} dataset. The combined \emch{}+\mech channel 
of the 0-jet and 1-jet bins are used for this measurement. It is found that
\begin{equation}
	\sigma^{\text{fid,0j}}_{\text{ggF}} &= \unit{$31.0^{+7.6}_{-7.3}$}{\femto\barn} \nonumber \\
	\sigma^{\text{fid,1j}}_{\text{ggF}} &= \unit{$9.9^{+4.0}_{-3.9}$}{\femto\barn} \nonumber
\end{equation}

% measured total cross section
% split mu into jet bins
% measured fiducial cross sections



