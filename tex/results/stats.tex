%!TEX root = ../../thesis.tex

A statistical model is built within the \histfactory software package \cite{HistFactory}, 
incorporating the various data-driven techniques described in Chapters~\ref{chap:ww} and 
\ref{chap:backgrounds}. The experimental data and MC-modelled expectations found within each 
signal and control region allow the model to be constrained, and for various hypotheses to 
be tested.



\subsection{Discriminant observables}
\label{sec:stat:sr_binning}

The fitting procedure discriminates between the signal and background processes through the 
distribution of certain sensitive observables. However, the degree to which this information 
can be exploited is limited by statistical uncertainties. The sensitivity of the fitting 
procedure can be optimised through the choice of discriminating observables, the number of 
bins in each observable, and the location of the bin boundaries.

The analysis features eight signal regions, defined by the criteria in 
\Section~\ref{sec:selection}: $\braces{\emch,\mech,\eech{}+\mmch} \otimes 
\braces{\text{0-jet,1-jet}}~~\oplus~~\braces{\emch,\mech} \otimes \braces{\twojet}$.
A three-dimensional fit of \mt, \ptsubleadlep and \mll is performed in the 
$\braces{\emch,\mech} \otimes \braces{\text{0-jet,1-jet}}$ channels, whereas a 
one-dimensional \mt fit is used in the others.

Additionally, an optimisation algorithm is used to determine the \mt bin boundaries in the 
$\braces{\emch,\mech,\eech{}+\mmch} \otimes \braces{\text{0-jet,1-jet}}$ signal regions. 
It is applied independently in each \ptsubleadlep-\mll bin in the \emch/\mech channels. 
This algorithm starts with bin boundaries that give an equal number of expected signal 
events in each bin. Then, each of the odd bin boundaries is scanned across \mt in order to 
optimise the sum in quadrature of the per-bin significances of the two adjacent bins
\begin{equation}
	Z_b = \sqrt{\left(N_{b}^{\text{sig}} \middle/ \sqrt{N_{b}^{\text{bkg}}}\right)^{\!\!2} + \left(N_{b+1}^{\text{sig}} \middle/ \sqrt{N_{b+1}^{\text{bkg}}}\right)^{\!\!2}}
\end{equation}
where $b$ is the bin number. The new bin boundary is accepted if the increase in $Z_b$ is 
statistically significant, \ie $(Z_b' - Z_b)/\Delta(Z_b' - Z_b) > 5$. This process repeats 
on the even bin boundaries, and continues to iterate between odd and even boundaries until 
their positions stabilise.

The binning of each discriminant observable is shown in \Table~\ref{tab:stat:sr_binning}.

\begin{table}[t]
	\begin{tabular}{cc@{\hskip 0.3in}ccc@{\hskip 0.3in}c}
		\toprule
		\multirow{2}{*}{\njets} & \multirow{2}{*}{Channel} & \multicolumn{3}{c}{Bin boundaries (\GeV)} & \multirow{2}{*}{$N_{\text{bins}}$} \\
		& & \ptsubleadlep & \mll & \mt & \\
		\midrule
		0-jet & \emch, \mech   & \hardrange{10,15,20,\infty} & \hardrange{10,30,55} & 10 bins: optimised & 60 \\
		0-jet & \eech{}+\mmch  & \hardrange{10,\infty} & \hardrange{12,55} & 10 bins: optimised & 10 \\
		1-jet & \emch, \mech   & \hardrange{10,15,20,\infty} & \hardrange{10,30,55} & \phantom{1}6 bins: optimised & 36 \\
		1-jet & \eech{}+\mmch  & \hardrange{10,\infty} & \hardrange{12,55} & \phantom{1}6 bins: optimised & 6 \\
		\twojet & \emch, \mech & \hardrange{10,\infty} & \hardrange{10,55} & \hardrange{0,50,80,130,\infty} & 4 \\
		\bottomrule
	\end{tabular}
	\caption{The binning of each discriminant for each signal region. The optimisation 
	algorithm used to determine the \mt bin boundaries is described in the text.}
	\label{tab:stat:sr_binning}
\end{table}



\subsection{Likelihood function}
\label{sec:stat:likelihood}

The statistical model describing the expected outcome of the experiment depends upon a set 
of parameters $\bvec{\alpha} = \parenths{\mu, \bvec{\theta}}$, where $\mu$ is the signal 
strength (a parameter of interest) and \bvec{\theta} is the set of nuisance parameters (\eg 
trigger efficiency, \WW extrapolation $\alpha_{\WW}$). The Higgs boson mass \mH could also 
be treated as a parameter of interest, but in practice this parameter is scanned during 
hypothesis testing.

The likelihood function expresses how likely a set of parameter values are, given that a 
particular dataset is observed. It is defined as the probability of producing the observed 
dataset, when the parameter values are input into the statistical model
\begin{equation}
	\mathcal{L}\parenths{\mu, \bvec{\theta}} &= \mathcal{L}\parenths{\mu, \bvec{\theta} ~\given~ \bvec{\mathcal{D}^{\text{SR}}}, \bvec{\mathcal{D}^{\text{CR}}}} \\
	&= f\parenths{\bvec{\mathcal{D}^{\text{SR}}}, \bvec{\mathcal{D}^{\text{CR}}}; \mu, \bvec{\theta}}
\end{equation}
where \bvec{\mathcal{D}^{\text{SR}}} and \bvec{\mathcal{D}^{\text{CR}}} are the sets of 
observed data in the signal regions (SRs) and control regions (CRs), respectively.
This can be decomposed into a product of probabilities
\begin{equation}
	\mathcal{L}\parenths{\mu, \bvec{\theta}} = \!\!\!\!\!\prod\limits_{c\in\text{channels}} \!\!\!\!\!f\parenths{\mathcal{D}_{c}^{\text{SR}}; \mu, \bvec{\theta}} ~\!\!\!\!\!\prod\limits_{c\in\text{channels}}\!\!\!\!\! f\parenths{\mathcal{D}_{c}^{\text{CR}}; \mu, \bvec{\theta}} ~\prod\limits_{i} ~f_i(\tilde{\theta}_i; \theta_i, \Delta\tilde{\theta}_i)
\end{equation}
where the SR channels are the eight SRs described in \Section~\ref{sec:stat:sr_binning}, and 
the CR channels include the data used in data-driven background estimations. Each 
$f(\tilde{\theta}; \theta, \Delta\tilde{\theta})$ acts to constrain the corresponding 
nuisance parameter, and is often a Gaussian distribution. The nominal value $\tilde{\theta}$ 
and uncertainty $\Delta\tilde{\theta}$ are determined by an auxiliary measurement, either 
experimental (\eg trigger efficiency) or theoretical (\eg $\alpha_{\WW}$).

The probability of producing an observed distribution $\mathcal{D}_c$ is described by 
a product of Poisson distributions
\begin{equation}
	f\parenths{\mathcal{D}_c; \mu, \bvec{\theta}} &= \!\prod\limits_{b\in\text{bins}}\! \Pois{N_{cb}^{\text{obs}}; \mu N_{cb}^{\text{sig}}\parenths{\bvec{\theta}} + N_{cb}^{\text{bkg}}\parenths{\bvec{\theta}}}
\end{equation}
where $\Pois{x; \lambda} = \lambda^x \eexp{-\lambda} / x!$, $N_{cb}^{\text{obs}}$ is the 
number of observed events, $N_{cb}^{\text{sig}}\parenths{\bvec{\theta}}$ is the number of 
expected signal events for a Standard Model Higgs boson (estimated by MC), and 
$N_{cb}^{\text{bkg}}\parenths{\bvec{\theta}}$ is the number of expected background events 
(estimated by a combination of data and MC). The product is over the bins of the 
distribution $\mathcal{D}_c$, which is one- or three-dimensional in the case of the SRs, 
and zero-dimensional in most of the CRs (\ie a single bin).

The expected numbers of events depend upon the nuisance parameters according to
\begin{equation}
	N_{cb}^{\text{sig}}\parenths{\bvec{\theta}} &= N_{cb}^{\text{sig}} ~\prod\limits_{i} \nu_{cbi}\parenths{\theta_i} \\
	N_{cb}^{\text{bkg}}\parenths{\bvec{\theta}} &= \!\!\!\!\!\sum\limits_{p\in\text{processes}}\!\!\!\!\! N_{cbp}^{\text{bkg}} ~\prod\limits_{i} \nu_{cbpi}\parenths{\theta_i}
\end{equation}
where $N = N(\tilde{\bvec{\theta}})$ is the expected events with the nominal nuisance 
parameters determined by auxiliary measurements, and $\nu_i(\theta_i)$ factorises the 
dependence upon the nuisance parameter $\theta_i$. The $\nu_i(\theta_i)$ are often 
determined by evaluating $\nu_i(\tilde{\theta}_i + \Delta\tilde{\theta}_i)$ and 
$\nu_i(\tilde{\theta}_i - \Delta\tilde{\theta}_i)$, and then interpolating and extrapolating 
to other values of $\theta_i$. Some nuisance parameters affect all processes (\eg trigger 
efficiency), whilst others only affect a particular process (\eg $\alpha_{\WW}$). 
Correlations exist between bins and channels, \eg a normalisation uncertainty would be 100\% 
correlated between bins. Statistical uncertainties in the MC samples are implemented as 
uncorrelated uncertainties in the total expected events in each bin, where the corresponding 
constraint $f(\tilde{\theta}; \theta, \Delta\tilde{\theta})$ is a Poisson distribution.



\subsection{Hypothesis testing}
\label{sec:stat:tests}

% nps constrained in the fit (pulls wrt auxiliary measurements)

\cite{Cowan:2010}
\cite{Junk:CLs,Read:CLs}

profile likelihood ratio:
\begin{equation}
	\tilde{\lambda}\parenths{\mu} = 
	\begin{cases}
		\frac{\mathcal{L}(\mu, \hat{\hat{\bvec{\theta}}}\parenths{\mu})}{\mathcal{L}(\hat{\mu}, \hat{\bvec{\theta}})} & \text{if~} \hat{\mu} \geq 0 \\
		\frac{\mathcal{L}(\mu, \hat{\hat{\bvec{\theta}}}\parenths{\mu})}{\mathcal{L}(0, \hat{\hat{\bvec{\theta}}}\parenths{0})} & \text{if~} \hat{\mu} < 0
	\end{cases}
\end{equation}

test statistic for measurement:
\begin{equation}
	\tilde{t}_{\mu} = -2 \ln \tilde{\lambda}\parenths{\mu}
\end{equation}

test statistic for discovery:
\begin{equation}
	q_0 = -2 \ln \tilde{\lambda}\parenths{0}
\end{equation}

test statistic for upper limits (leads to CLs):
\begin{equation}
	\tilde{q}_{\mu} = 
	\begin{cases}
		-2 \ln \tilde{\lambda}\parenths{\mu} & \text{if~} \hat{\mu} \leq \mu \\
		0 & \text{if~} \hat{\mu} > \mu
	\end{cases}
\end{equation}

% profile likelihood ratio (Neyman-Pearson lemma not actually for models with multiple parameters)
% profile likelihood ratio test statistic
% p-value
% CLs
% expected (Asimov data)

