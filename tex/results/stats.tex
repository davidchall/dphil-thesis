%!TEX root = ../../thesis.tex

A statistical model is built within the \histfactory software package \cite{HistFactory}, 
incorporating the various data-driven techniques described in Chapters~\ref{chap:ww} and 
\ref{chap:backgrounds}. The experimental data and MC-modelled expectations found within each 
signal and control region allow the model to be constrained, and for various hypotheses to 
be tested.



\subsection{Discriminant observables}
\label{sec:stat:sr_binning}

The fitting procedure discriminates between the signal and background processes through the 
distribution of certain sensitive observables. However, the degree to which this information 
can be exploited is limited by statistical uncertainties. The sensitivity of the fitting 
procedure can be optimised through the choice of discriminating observables, the number of 
bins in each observable, and the location of the bin boundaries.

The analysis features eight signal regions, defined by the criteria in 
\Section~\ref{sec:selection}: $\braces{\emch,\mech,\eech{}+\mmch} \otimes 
\braces{\text{0-jet,1-jet}}~~\oplus~~\braces{\emch,\mech} \otimes \braces{\twojet}$.
A three-dimensional fit of \mt, \ptsubleadlep and \mll is performed in the 
$\braces{\emch,\mech} \otimes \braces{\text{0-jet,1-jet}}$ channels, whereas a 
one-dimensional \mt fit is used in the others.

Additionally, an optimisation algorithm is used to determine the \mt bin boundaries in the 
$\braces{\emch,\mech,\eech{}+\mmch} \otimes \braces{\text{0-jet,1-jet}}$ signal regions. 
It is applied independently in each \ptsubleadlep-\mll bin in the \emch/\mech channels. 
This algorithm starts with bin boundaries that give an equal number of expected signal 
events in each bin. Then, each of the odd bin boundaries is scanned across \mt in order to 
optimise the sum in quadrature of the per-bin significances of the two adjacent bins
\begin{equation}
	Z_b = \sqrt{\left(N_{b}^{\text{sig}} \middle/ \sqrt{N_{b}^{\text{bkg}}}\right)^{\!\!2} + \left(N_{b+1}^{\text{sig}} \middle/ \sqrt{N_{b+1}^{\text{bkg}}}\right)^{\!\!2}}
\end{equation}
where $b$ is the bin number. The new bin boundary is accepted if the increase in $Z_b$ is 
statistically significant, \ie $(Z_b' - Z_b)/\Delta(Z_b' - Z_b) > 5$. This process repeats 
on the even bin boundaries, and continues to iterate between odd and even boundaries until 
their positions stabilise.

The binning of each discriminant observable is shown in \Table~\ref{tab:stat:sr_binning}.


\begin{table}[t]
	\begin{tabular}{cc@{\hskip 0.3in}ccc@{\hskip 0.3in}c}
		\toprule
		\multirow{2}{*}{\njets} & \multirow{2}{*}{Channel} & \multicolumn{3}{c}{Bin boundaries (\GeV)} & \multirow{2}{*}{$N_{\text{bins}}$} \\
		& & \ptsubleadlep & \mll & \mt & \\
		\midrule
		0-jet & \emch, \mech   & \hardrange{10,15,20,\infty} & \hardrange{10,30,55} & 10 bins: optimised & 60 \\
		0-jet & \eech{}+\mmch  & \hardrange{10,\infty} & \hardrange{12,55} & 10 bins: optimised & 10 \\
		1-jet & \emch, \mech   & \hardrange{10,15,20,\infty} & \hardrange{10,30,55} & \phantom{1}6 bins: optimised & 36 \\
		1-jet & \eech{}+\mmch  & \hardrange{10,\infty} & \hardrange{12,55} & \phantom{1}6 bins: optimised & 6 \\
		\twojet & \emch, \mech & \hardrange{10,\infty} & \hardrange{10,55} & \hardrange{0,50,80,130,\infty} & 4 \\
		\bottomrule
	\end{tabular}
	\caption{The binning of each discriminant for each signal region. The optimisation 
	algorithm used to determine the \mt bin boundaries is described in the text.}
	\label{tab:stat:sr_binning}
\end{table}



\subsection{Likelihood function}
\label{sec:stat:likelihood}

The fitting procedure involves the maximisation of a likelihood function, whose construction 
is now described.

\begin{equation}
	\mathcal{L}\parenths{\mu, \bvec{\theta}} &= \mathcal{L}\parenths{\mu, \bvec{\theta} ~\given~ \bvec{\mathcal{D}^{\text{SR}}}, \bvec{\mathcal{D}^{\text{CR}}}} \\
	&= f\parenths{\bvec{\mathcal{D}^{\text{SR}}}, \bvec{\mathcal{D}^{\text{CR}}}; \mu, \bvec{\theta}} \\
	&= \!\!\!\!\!\prod\limits_{c\in\text{channels}} \!\!\!\!\!f\parenths{\mathcal{D}_{c}^{\text{SR}}; \mu, \bvec{\theta}} ~\!\!\!\!\!\prod\limits_{c\in\text{channels}}\!\!\!\!\! f\parenths{\mathcal{D}_{c}^{\text{CR}}; \mu, \bvec{\theta}} ~\prod\limits_{i} ~f_i\parenths{\tilde{\theta}_i; \theta_i}
\end{equation}

\begin{equation}
	f\parenths{\mathcal{D}_c; \mu, \bvec{\theta}} &= \!\prod\limits_{b\in\text{bins}}\! \Pois{N_{cb}^{\text{obs}}; \mu N_{cb}^{\text{sig}}\parenths{\bvec{\theta}} + N_{cb}^{\text{bkg}}\parenths{\bvec{\theta}}} \\
	N_{cb}^{\text{sig}}\parenths{\bvec{\theta}} &= N_{cb}^{\text{sig}} ~\prod\limits_{i} \nu_{cbi}\parenths{\theta_i} \\
	N_{cb}^{\text{bkg}}\parenths{\bvec{\theta}} &= \!\!\!\!\!\sum\limits_{p\in\text{processes}}\!\!\!\!\! N_{cbp}^{\text{bkg}} ~\prod\limits_{i} \nu_{cbpi}\parenths{\theta_i}
\end{equation}

\begin{equation}
	\Pois{x; \lambda} = \frac{\lambda^x \eexp{-\lambda}}{x!}
\end{equation}


\cite{Cowan:2010}
\cite{Junk:CLs,Read:CLs}

% parameters of interest
% nuisance parameters constrained by auxiliary measurements




\subsection{Hypothesis testing}
\label{sec:stat:tests}

profile likelihood ratio:
\begin{equation}
	\tilde{\lambda}\parenths{\mu} = 
	\begin{cases}
		\frac{\mathcal{L}(\mu, \hat{\hat{\bvec{\theta}}}\parenths{\mu})}{\mathcal{L}(\hat{\mu}, \hat{\bvec{\theta}})} & \text{if~} \hat{\mu} \geq 0 \\
		\frac{\mathcal{L}(\mu, \hat{\hat{\bvec{\theta}}}\parenths{\mu})}{\mathcal{L}(0, \hat{\hat{\bvec{\theta}}}\parenths{0})} & \text{if~} \hat{\mu} < 0
	\end{cases}
\end{equation}

test statistic for measurement:
\begin{equation}
	\tilde{t}_{\mu} = -2 \ln \tilde{\lambda}\parenths{\mu}
\end{equation}

test statistic for discovery:
\begin{equation}
	q_0 = -2 \ln \tilde{\lambda}\parenths{0}
\end{equation}

test statistic for upper limits (leads to CLs):
\begin{equation}
	\tilde{q}_{\mu} = 
	\begin{cases}
		-2 \ln \tilde{\lambda}\parenths{\mu} & \text{if~} \hat{\mu} \leq \mu \\
		0 & \text{if~} \hat{\mu} > \mu
	\end{cases}
\end{equation}

% profile likelihood ratio (Neyman-Pearson lemma not actually for models with multiple parameters)
% profile likelihood ratio test statistic
% p-value
% CLs
% expected (Asimov data)

