%!TEX root = ../../thesis.tex

The analysis presented in this thesis has found significant experimental evidence for the 
process \ggHWWlvlv. However, it was on 4th July 2012 that the ATLAS and CMS collaborations 
announced the discovery of a new particle consistent with the Higgs boson of the Standard 
Model (SM), using a combination of decay channels to reject the null hypothesis with more 
than $5\sigma$ significance \cite{ATLAS-discovery,CMS-discovery}. Further \pp collisions 
were recorded until the end of December 2012, and then the entire Run I dataset was used 
to study the new particle in detail.

\Section~\ref{sec:searches} summarises the most important measurements of this new particle, 
which confirm that it has the qualitative properties of the Higgs boson of the SM. The 
theoretical implications of the discovery are considered in \Section~\ref{sec:implications}. 
Finally, in \Section~\ref{sec:outlook}, the outlook of Higgs boson measurements is assessed 
in the short and long term future.
