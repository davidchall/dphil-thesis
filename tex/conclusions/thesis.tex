%!TEX root = ../../thesis.tex

This thesis has described the experimental search for the \ggHWWlvlv process of Higgs boson 
production and decay. It uses the LHC Run~I dataset of \pp collisions recorded by the ATLAS 
detector, which corresponds to an integrated luminosity of \unit{4.5}{\invfb} at 
\unit{$\sqrt{s} = 7$}{\TeV} and \unit{20.3}{\invfb} at \unit{$\sqrt{s} = 8$}{\TeV}. 
An excess of events is observed with a significance of 4.8 standard deviations ($4.8\sigma$), 
which is consistent with Higgs boson production. The significance is extended to $6.1\sigma$ 
when the vector boson fusion production process is included. According to the convention 
adopted by the particle physics community, this constitutes a first observation, or 
discovery, of this process. The observed resonance is found to be consistent with the Higgs 
boson of the Standard Model with \unit{$\mH = 125$}{\GeV}, as are results from other LHC 
search channels described in \Section~\ref{sec:searches}.

The best-fit signal strength at \unit{$\mH = 125$}{\GeV} is found to be 
$\hat{\mu} = 1.11 \pm 0.22$, in excellent agreement with the Standard Model expectation. With 
a precision of 20\%, this \HWW analysis is the most sensitive $\mu$ measurement of the LHC 
Run~I Higgs boson analyses \cite{ATLAS:Hgg:RunI,ATLAS:HZZ:RunI,CMS:Hgamgam,CMS:HZZ,CMS:HWW}. 
When Run~II of the LHC begins in 2015, it will take some years to achieve a similar precision.
First, it will take time to record a dataset of similar size at the higher centre-of-mass 
energy of \unit{$\sqrt{s} = 13\text{ -- }14$}{\TeV}. Second, it will take time to understand 
the detector performance in the more challenging pile-up environment expected at Run~II. 
Third, improvements to the background estimations shall be needed (\eg using same-sign events 
to model the normalisation \textit{and} shape of the non-\WW diboson background, in order to 
reduce the associated theoretical uncertainties). Finally, advances in theoretical 
calculations and MC event generators shall improve the estimation of processes with large 
theoretical uncertainties (\eg ggF and \WW).
