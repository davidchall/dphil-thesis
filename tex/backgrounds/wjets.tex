%!TEX root = ../../thesis.tex

\Wjets events contribute to the background when a jet is misidentified as a lepton, and 
dijet events contribute when two jets are misidentified as leptons. Although the fake 
rates are very low, these backgrounds are significant because the \Wjets and dijet cross 
sections are much larger than those of Higgs boson production. Moreover, since the fake 
rates are sensitive to effects that are difficult to accurately model (such as jet 
substructure and particle interactions with the detector), a data-driven \textit{fake 
factor method} is used to estimate these backgrounds.

As these backgrounds have large uncertainties, their suppression is critically important. 
The electron and muon selection criteria are chosen to be tighter at low \pt (see 
\Section~\ref{sec:objects}), in order to reduce the fake rates where these backgrounds 
are largest. The dijet background is additionally rejected by requiring significant \met 
and by the \unit{$\maxmtw > 50$}{\GeV} cut in the \emch/\mech channels of the 1-jet bin.



\subsection{The fake factor method}
\label{sec:wjets:fakefactor_method}

The fake factor method defines two new reconstructed objects: anti-identified electrons 
\antiid{\Pe} and muons \antiid{\Pmu}, collectively known as anti-identified leptons 
\antiid{\Plepton}. Their loosened selection criteria ensures that they are highly 
contaminated by jets, while a veto on identified leptons \id{\Plepton} removes overlap 
between \id{\Plepton} and \antiid{\Plepton} objects (see \Section~\ref{sec:wjets:antiid}).
The fake factor $f_{\Plepton}$ is then defined as the ratio of efficiencies (\ie fake 
rates) for jets passing the \id{\Plepton} criteria to passing the \antiid{\Plepton} 
criteria.

In the following discussion, a \textit{sample} shall refer to a collection of events 
featuring a specific set of objects, \eg the signal sample 
$\mathcal{N}_{\id{\Plepton}\id{\Plepton}\text{,OS}}$ refers to \id{\Plepton}\id{\Plepton} 
events with opposite-sign charges (the \HWW analysis uses four signal samples: 
\id{\Plepton}\id{\Plepton} = \id{\Pe}\id{\Pe}, \id{\Pmu}\id{\Pmu}, \id{\Pe}\id{\Pmu} and 
\id{\Pmu}\id{\Pe}). A \textit{region} shall refer to events whose objects satisfy a set 
of criteria (also known as cuts), \eg the signal region refers to events passing the 
criteria in \Table~\ref{tab:event_selection}.

A dilepton sample $\mathcal{N}_{\id{\Plepton}\id{\Plepton}}$, a \Wjets control sample 
$\mathcal{N}_{\id{\Plepton}\antiid{\Plepton}}$ and a dijet control sample 
$\mathcal{N}_{\antiid{\Plepton}\antiid{\Plepton}}$ are defined, each with opposite-sign 
(OS) and same-sign (SS) charge variants. Note that $\mathcal{N}$ indicates a sample to 
which cuts may be applied, and anti-ID leptons are treated as leptons in such cuts when 
appropriate. Each sample has contributions from \Wjets and dijet events, and events with 
prompt leptons from the hard scatter (labelled EW):
\begin{equation}
	\mathcal{N}_{\id{\Plepton}\id{\Plepton},i} &= \mathcal{N}_{\id{\Plepton}\id{\Plepton},i}^{\text{\Wjets}} + \mathcal{N}_{\id{\Plepton}\id{\Plepton},i}^{\text{dijet}} + \mathcal{N}_{\id{\Plepton}\id{\Plepton},i}^{\text{EW}} \\
	\mathcal{N}_{\id{\Plepton}\antiid{\Plepton},i} &= \mathcal{N}_{\id{\Plepton}\antiid{\Plepton},i}^{\text{\Wjets}} + \mathcal{N}_{\id{\Plepton}\antiid{\Plepton},i}^{\text{dijet}} + \mathcal{N}_{\id{\Plepton}\antiid{\Plepton},i}^{\text{EW}} \\
	\mathcal{N}_{\antiid{\Plepton}\antiid{\Plepton},i} &= \mathcal{N}_{\antiid{\Plepton}\antiid{\Plepton},i}^{\text{\Wjets}} + \mathcal{N}_{\antiid{\Plepton}\antiid{\Plepton},i}^{\text{dijet}} + \mathcal{N}_{\antiid{\Plepton}\antiid{\Plepton},i}^{\text{EW}}
\end{equation}
where $i = \text{OS, SS}$. $\mathcal{N}_{\id{\Plepton}\id{\Plepton},i}$ is dominated by 
$\mathcal{N}_{\id{\Plepton}\id{\Plepton},i}^{\text{EW}}$, 
$\mathcal{N}_{\id{\Plepton}\antiid{\Plepton},i}$ is dominated by 
$\mathcal{N}_{\id{\Plepton}\antiid{\Plepton},i}^{\text{\Wjets}}$ and 
$\mathcal{N}_{\antiid{\Plepton}\antiid{\Plepton},i}$ is dominated by 
$\mathcal{N}_{\antiid{\Plepton}\antiid{\Plepton},i}^{\text{dijet}}$. 

The purpose of the fake factor method is to estimate the 
$\mathcal{N}_{\id{\Plepton}\id{\Plepton}\text{,OS}}^{\text{\Wjets}}$ and 
$\mathcal{N}_{\id{\Plepton}\id{\Plepton}\text{,OS}}^{\text{dijet}}$ contributions to the 
$\mathcal{N}_{\id{\Plepton}\id{\Plepton}\text{,OS}}$ signal sample. The SS estimations 
are also required for the non-\WW diboson background estimation (see 
\Section~\ref{sec:diboson}). It does this in a data-driven way by using the \Wjets and 
dijet control samples, subtracting the expected contaminations, and then multiplying by a 
data-driven fake factor:
\begin{equation}
	\mathcal{N}_{\id{\Plepton}\id{\Plepton},i}^{\text{pred,dijet}} &= \parenths{\mathcal{N}_{\antiid{\Plepton}\antiid{\Plepton},i}^{\text{data}} - \mathcal{N}_{\antiid{\Plepton}\antiid{\Plepton},i}^{\text{MC,\Wjets}} - \mathcal{N}_{\antiid{\Plepton}\antiid{\Plepton},i}^{\text{MC,EW}}} \cdot f_{\Plepton \vert \id{\Plepton}}^{\text{pred,dijet}} \cdot f_{\Plepton \vert \antiid{\Plepton}}^{\text{pred,dijet}} \\
	\mathcal{N}_{\id{\Plepton}\antiid{\Plepton},i}^{\text{pred,dijet}} &= \parenths{\mathcal{N}_{\antiid{\Plepton}\antiid{\Plepton},i}^{\text{data}} - \mathcal{N}_{\antiid{\Plepton}\antiid{\Plepton},i}^{\text{MC,\Wjets}} - \mathcal{N}_{\antiid{\Plepton}\antiid{\Plepton},i}^{\text{MC,EW}}} \cdot f_{\Plepton \vert \antiid{\Plepton}}^{\text{pred,dijet}} \\
	\mathcal{N}_{\id{\Plepton}\id{\Plepton},i}^{\text{pred,\Wjets}} &= \parenths{\mathcal{N}_{\id{\Plepton}\antiid{\Plepton},i}^{\text{data}} - \mathcal{N}_{\id{\Plepton}\antiid{\Plepton},i}^{\text{pred,dijet}} - \mathcal{N}_{\id{\Plepton}\antiid{\Plepton},i}^{\text{MC,EW}}} \cdot f_{\Plepton,i}^{\text{pred,\Wjets}}
\end{equation}
where $i = \text{OS, SS}$. It should be noted that the dijet contamination to the \Wjets 
control sample $\mathcal{N}_{\id{\Plepton}\antiid{\Plepton},i}^{\text{pred,dijet}}$ is 
data-driven from the dijet control sample. As discussed later, the fake factors are 
determined by the flavour composition of the jets, and thus depend upon the process. 
Also, $f_{\Plepton}^{\text{pred,dijet}}$ depends upon whether the other object in the 
event is an ID lepton ($f_{\Plepton \vert \id{\Plepton}}^{\text{pred,dijet}}$) or an 
anti-ID lepton ($f_{\Plepton \vert \antiid{\Plepton}}^{\text{pred,dijet}}$). This shall 
be discussed in \Section~\ref{sec:wjets:dijet_bkg}. Finally, note that 
$f_{\Plepton}^{\text{pred,\Wjets}}$ depends upon whether the event is OS or SS. This 
shall be discussed in \Section~\ref{sec:wjets:wjet_bkg}.

The rest of this section on the \Wjets and dijet backgrounds shall be spent describing 
the anti-identification lepton selection criteria (\Section~\ref{sec:wjets:antiid}), the 
measurement of fake factors in experimental data (Sections~\ref{sec:wjets:dijet_ff} and 
\ref{sec:wjets:zjet_ff}), and MC-based corrections to the fake factors to improve the 
background estimations (Sections~\ref{sec:wjets:dijet_bkg} and \ref{sec:wjets:wjet_bkg}).



\subsection{Lepton anti-identification criteria}
\label{sec:wjets:antiid}

With respect to the lepton reconstruction described in \Section~\ref{sec:objects}, the 
selection criteria of anti-ID leptons is looser in order to accept a large number of 
jets. An explicit veto upon fully identified leptons avoids overlap between samples. 
The fake rate of the anti-ID electrons is much higher than that of anti-ID muons, and 
consequently $f_{\Pe} \ll f_{\Pmu}$.

Relative to \Section~\ref{sec:objects:electrons}, anti-ID electrons of any \pt must fail 
the \textit{medium} identification criteria (though instead must have 
$n_{\text{pixel}}^{\text{hit}} + n_{\text{SCT}}^{\text{hit}} \geq 4$). Also, the tracker 
and calorimeter isolation are loosened to $\ptcone{0.3}/\et < 0.16$ and 
$\etcone{0.3}/\et < 0.30$.

Relative to \Section~\ref{sec:objects:muons}, anti-ID muons have their transverse impact 
parameter $d_0$ requirement removed. Also, the tracker isolation is removed and the 
calorimeter isolation is loosened to $\etcone{0.3}/\pt < 0.15$ for 
\unit{$\pt \in \hardrange{10,15}$}{\GeV}, $\etcone{0.3}/\pt < 0.25$ for 
\unit{$\pt \in \hardrange{15,20}$}{\GeV} and $\etcone{0.3}/\pt < 0.30$ for 
\unit{$\pt > 20$}{\GeV}.





\subsection{Dijet fake factor measurement}
\label{sec:wjets:dijet_ff}

\textit{In situ} fake factor measurements are made using dijet events. This involves 
counting the numbers of ID and anti-ID leptons in a dijet control region, subtracting the 
expected contamination from \PW and \PZ events, and calculating their ratio. 
$f_{\Plepton}^{\text{data,dijet}}$ is measured as a function of \pt and $\eta$.

Following data quality requirements, events are selected using very loose (no isolation 
or electron identification criteria), but highly prescaled, lepton triggers. To reduce 
the effects of prescaling, different triggers were used for different \pt ranges, and a 
trigger containing electron identification criteria was added to aid measurement of the 
numerator. 
In the $f_{\Pe}$ measurement, the \verb|EF_e5_etcut| (\unit{0.012}{\invpb}) and 
\verb|EF_e5_medium1| (\unit{0.24}{\invpb}) triggers were used for \unit{$\pt < 20$}{\GeV} 
and the \verb|EF_g24_etcut| (\unit{2.1}{\invpb}) trigger was used for 
\unit{$\pt > 20$}{\GeV}. 
In the $f_{\Pmu}$ measurement, the \verb|EF_mu6| (\unit{0.94}{\invpb}) trigger was used 
for \unit{$\pt < 15$}{\GeV} and the \verb|EF_mu15| (\unit{23}{\invpb}) trigger was used 
for \unit{$\pt > 15$}{\GeV}. The trigger naming scheme is explained in the caption of 
\Table~\ref{tab:sel:triggers}.

The dijet control region requires events to have a jet with \unit{$\pt > 15$}{\GeV} (see 
\Section~\ref{sec:objects:jets} for jet selection) balancing the triggered lepton object,
$\Delta\phi\parenths{\Plepton,j} < 0.7$. To suppress contamination from \PW events we 
require \unit{$\mtw < 30$}{\GeV}, and to suppress the \PZ background we veto events with 
a lepton pair satisfying \unit{$\mods{\mll - \mZ} < 13$}{\GeV}. Note that these 
criteria apply to both \id{\Plepton} and \antiid{\Plepton} objects. Normalisation factors 
for the residual \PW and \PZ backgrounds are derived by inverting the respective veto.

The measured electron and muon fake factors are shown in \Figure~\ref{fig:wjets:ff_data}. 
The uncertainty is dominated by uncertainties in the subtracted contamination (which is 
conservatively scaled up and down by 20\%). $f_{\Plepton}^{\text{data,dijet}}$ is used in 
the dijet background estimation, as described in \Section~\ref{sec:wjets:dijet_bkg}.

\begin{figure}
	\includegraphics[width=0.495\textwidth]{custom_images/wjets/ff_el_data}
	\hfill
	\includegraphics[width=0.495\textwidth]{custom_images/wjets/ff_mu_data}
	\caption{The fake factor measured in dijet (red) and \Zjets (blue) events versus \pt, 
	for electrons (left) and muons (right). The error bars include statistical 
	uncertainties and uncertainties in the background subtraction.}
	\label{fig:wjets:ff_data}
\end{figure}



\subsection{\Zjets fake factor measurement}
\label{sec:wjets:zjet_ff}

\textit{In situ} fake factor measurements are also made using \Zjets events. This 
involves counting the numbers of ID and anti-ID leptons in a \Zjets control region (CR), 
subtracting the expected electroweak contamination (\Zgamma, \ZZ, \Zgstar, \WZ, \Wgstar), 
and calculating their ratio.

Following data quality requirements, events are selected using unprescaled lepton 
triggers. This is possible because the triggered object is a lepton from the \PZ boson 
decay, and the \pt threshold can therefore be relatively high. In the $f_{\Pe}$ 
measurement, the \verb|EF_e24vhi_medium1| and \verb|EF_e60_medium1| triggers supported 
\unit{$\pt > 25$}{\GeV}. In the $f_{\Pmu}$ measurement, the \verb|EF_mu24i_tight| and 
\verb|EF_mu36_tight| triggers were used with the dilepton \verb|EF_mu18_tight_mu8_EFFS| 
trigger to support \unit{$\pt > 22$}{\GeV}. The trigger naming scheme is explained in the 
caption of \Table~\ref{tab:sel:triggers}.

The \Zjets CR requires a pair of same-flavour and oppositely charged ID 
leptons to reconstruct the \PZ boson mass, \unit{$81 < \mll < 107$}{\GeV}. To suppress 
the \ZZ background we veto events with another lepton pair satisfying 
\unit{$76 < \mll < 107$}{\GeV}, and to suppress \WZ contamination we require 
\unit{$\mtw < 30$}{\GeV}. Residual diboson backgrounds (\Zgamma, \ZZ, \Zgstar, \WZ, 
\Wgstar) are subtracted using predictions from MC simulation.

The measured electron and muon fake factors are shown in \Figure~\ref{fig:wjets:ff_data}. 
The uncertainty is dominated by statistical uncertainty. For this reason, 
$f_{\Plepton}^{\text{data,\Zjets}}$ is measured as a function of \pt only, and the $\eta$ 
dependence is injected from $f_{\Plepton}^{\text{data,dijet}}$. The uncertainty due to 
electroweak subtraction is also significant, because the contamination to the \Zjets CR 
is not negligible. Although $f_{\Plepton}^{\text{data,\Zjets}}$ has larger uncertainties 
than $f_{\Plepton}^{\text{data,dijet}}$, it shall be used in the \Wjets background 
estimation. This is because jets in \Zjets and \Wjets events are expected to have similar 
flavour composition, and therefore similar fake factors (see 
\Section~\ref{sec:wjets:wjet_bkg}).



\subsection{Dijet background estimation}
\label{sec:wjets:dijet_bkg}

\begin{equation}
	f_{\Plepton \vert \antiid{\Plepton}}^{\text{pred,dijet}}\parenths{\pt,\eta} &= f_{\Plepton}^{\text{data,dijet}}\parenths{\pt,\eta} \cdot \frac{f_{\Plepton \vert \antiid{\Plepton}}^{\text{MC,dijet}}}{f_{\Plepton \vert j}^{\text{MC,dijet}}} \\
	f_{\Plepton \vert \id{\Plepton}}^{\text{pred,dijet}}\parenths{\pt,\eta} &= f_{\Plepton}^{\text{data,dijet}}\parenths{\pt,\eta} \cdot \frac{f_{\Plepton \vert \id{\Plepton}}^{\text{MC,dijet}}}{f_{\Plepton \vert j}^{\text{MC,dijet}}}
\end{equation}




\subsection{\Wjets background estimation}
\label{sec:wjets:wjet_bkg}

\begin{equation}
	f_{\Plepton,i}^{\text{pred,\Wjets}}\parenths{\pt,\eta} &= f_{\Plepton}^{\text{data,\Zjets}}\parenths{\pt} \cdot \frac{f_{\Plepton}^{\text{data,dijet}}\parenths{\pt, \eta}}{f_{\Plepton}^{\text{data,dijet}}\parenths{\pt}} \cdot \frac{f_{\Plepton,i}^{\text{MC,\Wjets}}}{f_{\Plepton}^{\text{MC,\Zjets}}}
\end{equation}


