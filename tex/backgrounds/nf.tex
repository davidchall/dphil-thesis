%!TEX root = ../../thesis.tex

It is helpful to express the data-driven estimations of several backgrounds in terms of a 
normalisation factor, which is the ratio of the data-driven predicted yield to the MC-only 
predicted yield. Their pre-fit and post-fit values are summarised in 
\Table~\ref{tab:bkg:NFs}, where the fit can change the values due to the pull of nuisance 
parameters (see \Section~\ref{sec:stat:likelihood}).\todo{Update NFs}

These can also be used to improve the background estimations in regions other than the signal 
region; however, this does neglect theoretical uncertainties in the extrapolation from the 
corresponding control region. Plotted distributions in 
Chapters~\ref{chap:selection}--\ref{chap:backgrounds} use pre-fit normalisation factors in 
this way.

\begin{table}[t]
	\begin{tabularx}{0.75\textwidth}{l@{\hskip 0.3in}YYY}
		\toprule
		\multirow{2}{*}{Process} & \multicolumn{3}{c}{Pre-fit (post-fit) normalisation factor} \\
		& 0-jet & 1-jet & \twojet \\
		\midrule
		\WW             & 1.22 (1.21) & 1.08 (1.08) & -- \\
		Non-\WW diboson & 0.92 (0.96) & 0.96 (0.96) & -- \\
		Top             & 1.09 (1.09) & 1.02 (1.03) & 1.00 (1.00) \\
		\DYtt           & 1.00 (0.99) & 1.06 (1.05) & 0.96 (0.97) \\
		\bottomrule
	\end{tabularx}
	\caption{The data-driven normalisation factor used to scale the MC description of each 
	background process. These can change in the fit due to the pull of nuisance parameters 
	(see \Section~\ref{sec:stat:likelihood}). The \Wjets and dijet processes are fully 
	data-driven.}
	\label{tab:bkg:NFs}
\end{table}
