%!TEX root = ../../thesis.tex

In addition to the \WW background, several other processes contribute significant background. 
As they are often difficult to model accurately in the \HWW phase space, they are estimated 
by sophisticated data-driven methods. Even so, there is usually some underlying dependence 
upon MC, which is summarised in \Table~\ref{tab:bkg:mc_samples}.

This chapter describes the estimation of non-\WW backgrounds: \Wjets and dijet in 
\Section~\ref{sec:wjets}, non-\WW diboson in \Section~\ref{sec:diboson}, top in 
\Section~\ref{sec:top}, and \DY in \Section~\ref{sec:dy}.

\begin{table}[b]
	\begin{tabular}{c@{\hskip 0.3in}c}
		\toprule
		Process & MC generator (\twojet bin) \\
		\midrule
		\WW        & \meps{\powhegbox}{\pythia{6}}, \meps{\ggtoww}{\fherwig} (\sherpa) \\
		top        & \meps{\powhegbox}{\pythia{6}}, \meps{\acermc}{\pythia{6}} \\
		\Wjets, \DY, \Wgamma & \meps{\alpgen}{\fherwig} \\
		\WZ        & \meps{\powhegbox}{\pythia{8}} \\
		\ZZ        & \meps{\powhegbox}{\pythia{8}}, \meps{\ggtozz}{\fherwig} \\
		\Wgstar, \Zgstar, \Zgamma & \sherpa \\
		\bottomrule
	\end{tabular}
	\caption{MC generators used to model backgrounds to the \HWW search. These are used with 
	the data-driven techniques described in the text.}
	\label{tab:bkg:mc_samples}
\end{table}
