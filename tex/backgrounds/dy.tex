%!TEX root = ../../thesis.tex

The \DY background is naturally split into \DYll and \DYtt processes. The former 
contributes almost exclusively to the \eech/\mmch channels,\footnote{
	In rare cases, \DYll events can enter the \emch/\mech channels. For example 
	\HepProcess{\DY \HepTo \Pmu\Pmu\Pphoton}, where a muon radiates a photon which 
	subsequently converts and is reconstructed as an electron.
}
where it is the dominant background. The latter can contribute to any channel since the 
two \HepProcess{\Ptau \HepTo \Plepton\Pnulepton\Pnut} decays are independent, though is 
doubly suppressed by the small $\text{BR}\parenths{\HepProcess{\Ptau \HepTo 
\Plepton\Pnulepton\Pnut}} = 17.6\%$ \cite{PDG:2012}.

Although \DYll does not feature prompt neutrinos, degradation of the \met resolution due 
to high pile-up can cause some \DYll events to exhibit significant \met. Since this is a 
difficult effect to model and \DYll is the dominant background to the \eech/\mmch 
channels, a high threshold of \unit{$\metrel > 40$}{\GeV} is used in the pre-selection. 
This is further tightened by tracker-based \trackmetrel cuts. On the other hand, \DYtt 
does feature prompt neutrinos and is suppressed by other cuts, such as the \mtautau veto.

The \DY backgrounds are estimated by data-driven techniques, in combination with MC 
modelling provided by \meps{\alpgen}{\fherwig}. Overlap between the \DY MC and the 
\Zgamma MC is removed through careful consideration of the MC event records.



\subsection{\DY boson transverse momentum}
\label{sec:dy:pt}

Selecting \eech/\mmch events with \unit{$\mods{\mll - \mZ} < 15$}{\GeV} results in a very 
pure sample of \DYll events, enabling the MC to be validated. In doing so, the \ptll 
distribution is found to be poorly modelled at \unit{$\ptll > 30$}{\GeV} in the 0-jet 
bin (see \Figure~\ref{fig:dy:ptZ_reweight}), despite being well modelled inclusively. 
This is unsurprising since a difficult phase space, sensitive to soft hadronic activity 
and jet shapes, has been selected: requiring a highly boosted \DY boson whilst vetoing 
events with jets. It is important to model \ptll accurately, as other observables such as 
\dphill and \ptleadlep are correlated.

For this reason, a data-driven correction to the \ptZ distribution is employed. It is 
derived in \mmch 0-jet events with \unit{$\mods{\mll - \mZ} < 15$}{\GeV}, by 
comparing the \ptll distribution observed in experimental data to that predicted by 
uncorrected MC. Each 0-jet MC event is then weighted according to its hadron-level \ptZ. 
This is found to improve the modelling of detector-level observables such as \ptll, 
\dphill and \ptleadlep (see \Figure~\ref{fig:dy:ptZ_reweight}). This correction is 
applied to the \HepProcess{\DY \HepTo \Pe\Pe/\Pmu\Pmu/\Ptau\Ptau} processes, though only 
to the 0-jet bin.

\begin{figure}[p]
	\includegraphics[width=0.495\textwidth]{tex/motivation/sombrero_comical}
	\hfill
	\includegraphics[width=0.495\textwidth]{tex/motivation/sombrero_comical}
	\\
	\includegraphics[width=0.495\textwidth]{tex/motivation/sombrero_comical}
	\hfill
	\includegraphics[width=0.495\textwidth]{tex/motivation/sombrero_comical}
	\\
	\includegraphics[width=0.495\textwidth]{tex/motivation/sombrero_comical}
	\hfill
	\includegraphics[width=0.495\textwidth]{tex/motivation/sombrero_comical}
	\caption{Leptonic distributions in the 0-jet \PZ control region, before (left) and 
	after (right) the \ptZ correction is applied. The distributions shown are \ptll 
	(top), \dphill (middle) and \ptleadlep (bottom).}
	\label{fig:dy:ptZ_reweight}
\end{figure}

The \ptZ mismodelling might be different in the signal region to the \PZ control region 
where it is derived. The validity of this extrapolation of the correction was tested by 
deriving similar corrections to \sherpa (instead of experimental data). This indicated 
there is a correlation between the correction and the \met requirement. Thus, another 
correction is derived with an additional \unit{$\corrtrackmet > 20$}{\GeV} cut, which is 
used to estimate the uncertainty in the correction.



\subsection{\DYtt estimation}
\label{sec:dy:tautau}

The \DYtt background is measured in a dedicated control region (CR), and then 
extrapolated to the signal region (SR) using MC
\begin{equation}
	N_{\HepProcess{\PZ\HepTo\Ptau\Ptau}}^{\text{pred,SR}} &= \alpha_{\HepProcess{\PZ\HepTo\Ptau\Ptau}} \cdot \parenths{N^{\text{data,CR}} - N_{\text{non-}\HepProcess{\PZ\HepTo\Ptau\Ptau}}^{\text{pred,CR}}} \\
	\alpha_{\HepProcess{\PZ\HepTo\Ptau\Ptau}} &= N_{\HepProcess{\PZ\HepTo\Ptau\Ptau}}^{\text{MC,SR}} / N_{\HepProcess{\PZ\HepTo\Ptau\Ptau}}^{\text{MC,CR}} \,.
\end{equation}
A control region is defined for each jet bin in the \emch/\mech channels, as shown in 
\Table~\ref{tab:dytt_cr_sel}. These are used to extrapolate to the respective signal 
regions, for all channels.

\todo[inline]{Plot CR}

\begin{table}[t]
	\begin{tabularx}{0.7\textwidth}{YYY}
		\toprule
		\multicolumn{3}{c}{\emch/\mech} \\
		\midrule
		\multicolumn{3}{c}{$\ptleadlep > 22$ and $\ptsubleadlep > 10$} \\
		\multicolumn{3}{c}{$\mll > 12$} \\
		\multicolumn{3}{c}{$\corrtrackmet > 20$} \\
		\cmidrule(lr){1-3}
		0-jet bin & 1-jet bin & \twojet bin \\
		\cmidrule(lr){1-3}
		-- & $\nbjets = 0$ & $\nbjets = 0$ \\
		-- & $\maxmtw > 50$ & -- \\
		-- & $\mtautau > \mZ - 25$ & -- \\
		-- & -- & Fail CJV or OLV \\
		$\mll < 80$ & $\mll < 80$ & $\mll < 70$ \\
		$\dphill > 2.8$ & -- & $\dphill > 2.8$ \\
		\bottomrule
	\end{tabularx}
	\caption{Event selection criteria of the \DYtt control regions. Cuts on energy, 
	momentum and mass are given in \GeV, and angular cuts are given in radians. The 
	relevant variables are described in 
	\Chapter~\ref{chap:selection}.}
	\label{tab:dytt_cr_sel}
\end{table}

The corresponding normalisation factors are \stat{1.01}{0.02} in the 0-jet bin, 
\stat{1.08}{0.04} in the 1-jet bin, and \stat{1.05}{0.10} in the \twojet bin.



\subsection{\DYll estimation}
\label{sex:dy:ll}


