%!TEX root = ../../thesis.tex

Following the \HWW decay, each \PW boson will decay leptonically or hadronically, with 
$\text{BR}\parenths{\HepProcess{\PW\HepTo\Plepton\Pnu}} = 10.8\%$ and 
$\text{BR}\parenths{\HepProcess{\PW\HepTo \text{hadrons}}} = 67.6\%$ \cite{PDG:2012}. 
Thus, \textit{dileptonic}, \textit{semi-leptonic} and \textit{hadronic} final states are 
conceivable. Although the dileptonic channel is suppressed by BRs, it is ultimately 
the most sensitive as the other two have larger backgrounds. This chapter 
will describe the dileptonic search, and henceforth 
`lepton' and \Plepton shall refer to an electron or muon.\footnote{
	Events with one or two \HepProcess{\PW\HepTo\Ptau\Pnu} decays can 
	contribute to the dileptonic search when the \Ptau decays to an electron or muon. This 
	contribution is small however, since
	$\text{BR}\parenths{\HepProcess{\Ptau\HepTo\Plepton\Pnulepton\Pnut}} = 17.6\%$ 
	\cite{PDG:2012}. Also, the kinematics of such events are different due to the 
	additional decay(s) and neutrinos.
}

Electroweak fits favour a Higgs boson with mass $\mH < 2\mW$ (see \Figure~\ref{fig:ewfit}).
It is therefore important for the \HWW search to be sensitive to off-shell \PW bosons. 
Experimentally, this means using leptons with low \pt thresholds, which unfortunately 
have reduced purity due to large misidentified hadronic backgrounds.

Since neutrinos do not interact with the detector, it is only possible to infer their 
combined transverse momentum from an imbalance in the visible momenta, called \met. 
Thus, it is not possible to fully reconstruct a 
mass peak in the \HWWlvlv search, so to be sensitive to the Higgs boson it becomes 
crucial to accurately understand the many background processes.

The basic experimental signature is two oppositely charged leptons and significant \met. 
However, there are background processes that exhibit the same signature. Others have an 
aspect of the signature faked by mismeasurement, or some part of their final state is not 
reconstructed. \Table~\ref{tab:bkg_summary} introduces the different backgrounds. Jets 
are a convenient way to separate the contributions of different background processes.

Jets can also be used to separate the gluon-gluon fusion (ggF) and vector boson fusion 
(VBF) production modes of the Higgs boson (see \Section~\ref{sec:properties}). The search 
described below is designed for the ggF production mode, and this shall become 
particularly apparent when describing the \twojet bin in \Section~\ref{sec:selection:2j}.

\begin{table}[t]
	\begin{tabular}{l@{\hskip 0.3in}l}
		\toprule
		Background        & Mechanism of \HepProcess{\Plepton\Plepton + \met} signature \\
		\midrule
		\WW               & irreducible \\
		\ttbar, \HepProcess{\Ptop\PW} & irreducible \\
		\HepProcess{\Ptop\Pbottom}, \HepProcess{\Ptop\Pbottom\Pquark} & jet fakes lepton \\
		\DYll             & fake \met \\
		\DYtt             & irreducible \\
		\Wjets, dijet     & jet(s) fake lepton(s) \\
		\Wgamma           & photon fakes electron \\
		\WZ, \Wgstar, \ZZ & unreconstructed lepton(s) \\
		\bottomrule
	\end{tabular}
	\caption{Summary of how each background produces the 
	\HepProcess{\Plepton\Plepton + \met} experimental signature, inclusive in the number of 
	jets.}
	\label{tab:bkg_summary}
\end{table}
