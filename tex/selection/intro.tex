%!TEX root = ../../thesis.tex

The \WW decay of the Higgs boson is a promising search channel as it has a large \ac{BR} 
for a wide range of \mH. In fact, it is the most probable decay for \unit{$\mH > 
135$}{\GeV} (see \Figure~\ref{fig:higgs_br}). Each \PW boson decays with 
$\text{BR}\parenths{\HepProcess{\PW\HepTo\Plepton\Pnu}} = 10.8\%$ 
or $\text{BR}\parenths{\HepProcess{\PW\HepTo \text{hadrons}}} = 67.6\%$ \cite{PDG:2012}, 
and so \textit{dileptonic}, \textit{semi-leptonic} and \textit{hadronic} final states are 
conceivable. Although the dileptonic channel is suppressed by \acp{BR}, it is ultimately 
the most sensitive as the other two have large experimental backgrounds. This thesis shall 
describe the dileptonic search \HWWlvlv where $\Plepton = \Pe, \Pmu$ and henceforth 
`lepton' shall refer to an electron or muon.\footnote{
	Events with one or two \HepProcess{\PW\HepTo\Ptau\Pnu} decays can 
	contribute to the dileptonic search when the \Ptau decays to an electron or muon. This 
	contribution is small however, since
	$\text{BR}\parenths{\HepProcess{\Ptau\HepTo\Plepton\Pnulepton\Pnut}} = 17.6\%$ 
	\cite{PDG:2012}. The kinematics of such events can be different due to the additional 
	decay(s) and neutrinos.
}

Since neutrinos do not interact with the detector, it is only possible to infer their 
transverse momentum from an imbalance in the visible momenta, called \met. When there are 
multiple neutrinos, only the transverse component of the vector sum of the neutrino 
momenta can be inferred from the \met. Thus, it is not possible to fully reconstruct a 
mass peak in the \HWWlvlv search, and to be sensitive to the Higgs boson it becomes 
crucial to understand the many background processes.

The basic experimental signature is two oppositely charged leptons and significant \met. 
However, there are background processes that exhibit the same signature, such as \WW and 
\ttbar. There are others where some aspect has been faked by mismeasurement, for example a 
fake lepton in \Wjets and fake \met in \Zjets. Jets are a convenient way to separate the 
contributions of different background processes. They can also be used to separate the 
\ac{ggF} and \ac{VBF} production modes of the Higgs boson (see 
\Section~\ref{sec:properties}).

\Section~\ref{sec:objects} contains a description of how each physics object (electrons, 
muons, jets, \met) is reconstructed by the ATLAS detector. Then, 
\Section~\ref{sec:selection} outlines the criteria by which Higgs boson events are 
selected and background events are rejected. In describing these criteria the backgrounds 
shall be introduced, though their estimation shall be detailed in \Chapter~\ref{chap:ww} 
and \Chapter~\ref{chap:backgrounds}.
