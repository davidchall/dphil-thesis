%!TEX root = ../../thesis.tex

It is clear from (\ref{eq:event_rate}) that the luminosity delivered to the detector is a 
key input when studying \pp collisions at the \ac{LHC}. It is directly proportional to the 
expected number of events, and uncertainties in its value will be propagated to measured 
cross sections. The measurement of the luminosity delivered to the ATLAS detector is 
described in \Section~\ref{sec:dataset:lumi}, followed by a description of the dataset 
used in this thesis.



\subsection{Luminosity measurement}
\label{sec:dataset:lumi}

Beam losses incurred by the \pp collisions cause the luminosity to decay with time (a
typical \ac{LHC} run lasts \about 10 hours). It is therefore necessary to measure 
the instantaneous luminosity in real-time. The method is detailed in \cite{Lumi2011}, but 
is briefly outlined below.

At the \ac{LHC}, the number of inelastic \pp interactions per bunch crossing follows a 
Poisson distribution, with a mean value $\mu$. As mentioned in \Section~\ref{sec:lhc}, 
a large luminosity results in $\mu > 1$ (a condition known as pile-up). Thus, the 
luminosity $L$ can be monitored ``online'' by measuring the observed number of interactions
per crossing $\mu_{\text{vis}}$, using
\begin{equation}
	L = \frac{\mu n_b f_{\text{rev}}}{\sigma_{\text{inel}}}
	= \frac{\mu_{\text{vis}} n_b f_{\text{rev}}}{\sigma_{\text{vis}}}
	\label{eq:lumi_measure}
\end{equation}
where $n_b$ is the number of bunches per beam, $f_{\text{rev}}$ is the \ac{LHC} revolution 
frequency, and $\sigma_{\text{inel}}$ is the inelastic \pp cross section. The 
expression is rewritten in terms of ``visible'' quantities, owing to inefficiencies in the
detector and algorithm used to measure $\mu$.

The \ac{BCM} and LUCID detectors, respectively situated \unit{2}{\metre} and 
\unit{17}{\metre} down the beamline from the interaction point, each provide 
bunch-by-bunch luminosity measurements. The \ac{BCM} consists of sixteen small diamond 
sensors, and was primarily designed to issue beam-abort requests when beam losses risk 
damaging the ATLAS detector. Diamond was chosen instead of silicon due to its 
radiation-hardness and short time resolution (\unit{\about0.7}{\nano\second}).
LUCID consists of sixteen aluminium tubes of C$_4$F$_{10}$ gas, which radiate and collect 
Cherenkov photons when struck by charged particles.

\ac{BCM} and LUCID are calibrated during dedicated van der Meer (vdM) scans, effectively 
determining $\sigma_{\text{vis}}$ in (\ref{eq:lumi_measure}). In a vdM scan, event rates 
are measured while the beams are separated in steps of known distance, allowing direct 
measurement of beam sizes $\varSigma_x$ and $\varSigma_y$. The absolute luminosity is then 
determined through (\ref{eq:lumi_beam}). The uncertainty in the vdM calibration dominates 
the uncertainty in the delivered luminosity.



\subsection{Run I dataset}
\label{sec:dataset:dataset}

Data-taking operations during run I of the LHC were incredibly successful, with impressive 
improvements made to peak luminosity throughout this period. Some important parameters of 
the run I dataset are summarised in \Table~\ref{tab:dataset}, and the pile-up conditions are shown in \Figure~\ref{fig:pileup}. These show that a larger dataset was obtained in 2012 compared with 2011, but at the expense of a higher pile-up environment.

\begin{table}[h]
	\begin{tabular}{l@{\hskip 0.3in}c@{\hskip 0.3in}c@{\hskip 0.3in}c@{\hskip 0.3in}c}
	& 2010 & 2011 & 2012 & Design \\
	\hline
	Centre-of-mass energy (\TeV)         & 7 & 7 & 8 & 14 \\
	Minimum bunch spacing (\nano\second) & 150 & 50 & 50 & 25 \\
	Peak luminosity (\unit{$10^{33}$}{\lumiunits}) & 0.2 & 3.6 & 7.7 & 10 \\
	Delivered luminosity (\invfb)       & 0.047 & 5.46 & 22.8 & -- \\
	Recorded luminosity (\invfb)        & 0.044 & 5.08 & 21.3 & -- \\
	Analysed luminosity (\invfb)        & 0.000 & 4.53 & 20.7 & -- \\
	Luminosity uncertainty $\delta L/L$ & 3.5\% & 1.8\% & 2.8\% & -- \\
	\end{tabular}
	\caption{Summary of \pp collision data during \ac{LHC} run I. The integrated 
	luminosity analysed in the \HWW search is also shown. Luminosities use the offline 
	calibration.}
	\label{tab:dataset}
\end{table}

\begin{figure}[h]
	\includegraphics[width=\mediumfigwidth]{tex/experiment/pileup}
	\caption{Mean number of interactions per bunch crossing $\mu$ for the 2011 (blue) and 
	2012 (green) datasets, calculated with an inelastic \pp cross section of 
	\unit{71.5}{\milli\barn} at \unit{$\sqrt{s} = 7$}{\TeV} and 
	\unit{73.0}{\milli\barn} at \unit{$\sqrt{s} = 8$}{\TeV}.
	Luminosities use the online calibration.}
	\label{fig:pileup}
\end{figure}

