%!TEX root = ../thesis.tex

\begin{preface}

As a DPhil research topic, the \HWW analysis has proven to be a baptism of fire. It is the 
most complicated of the three ``discovery channels'',\footnote{
	The \HepProcess{\Pphoton\Pphoton}, \ZZ and \WW decay channels quickly gave sensitivity 
	to the Higgs boson ultimately discovered.
}
as it involves a variety of physics objects and requires a good understanding of many 
difficult backgrounds. As such, the analysis took huge effort from a large number of 
individuals. My role focussed on theoretical aspects of the signal and background 
modelling, and these parts shall be emphasised. I contributed to multiple iterations of 
the analysis \cite{HWW-Moriond:2012,HWW-MVA,HWW-7TeV,ATLAS-discovery,HWW-discovery-contribution,HWW-HCP,HWW-Moriond,ATLAS:combination:2013}, 
though the version presented here is unpublished at the time of writing \cite{HWW-RunI}. I 
also co-authored the third Yellow Report produced by the LHC Higgs Cross Section Working 
Group \cite{YR3}.

When I began the degree in October 2010, there was no direct evidence for a Higgs boson. 
This thesis is written from a personal perspective and motivates a low mass search by 
electroweak fits, when in fact this aspect was motivated later by observations of a 
resonance in the \HepProcess{\Pphoton\Pphoton} and \ZZ channels.\footnote{
	Dedicated high mass searches for \HWW have also been performed \cite{HWW-highmass}.
}
Also, an advanced search strategy is described, though the discovery of \HWW was actually 
a gradual process with multiple iterations of blinding, optimising and unblinding the 
analysis. As more data were recorded and the analysis was enhanced, the results improved.

Early on, I gained relevant insight by performing multiple \WW cross section measurements 
\cite{WW-35ipb,WW-EPS,WW-1ifb,WW-7TeV}. My main contribution was a jet veto 
correction factor applied to the \WW signal, which reduces the dominant uncertainty in the 
analysis. This measurement shall be described when considering the \WW background to the 
\HWW search.

To qualify for authorship within the ATLAS collaboration, I performed Run Control shifts. 
I also worked within the Versatile Link project \cite{VersatileLink} to investigate 
radiation hardened optical components for the HL-LHC. As this research does not easily 
relate to the Higgs boson, it is excluded from this thesis. However, I have published 
articles on the radiation tolerance of optical fibres \cite{VersatileLinkFibres} and their 
connectors \cite{VersatileLinkConnectors}.

\bibliographystyle{thesis}
\bibliography{theory,pheno,mc,experiment}

\end{preface}
