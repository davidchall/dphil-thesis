%!TEX root = ../../thesis.tex

We have seen in \Section~\ref{sec:mc} how coloured partons produced in a hard subprocess 
(in the \ac{ME}) or radiated from the incoming partons (\ac{ISR}) will each produce a 
shower of partons, which subsequently hadronise. By measuring the energy and direction of 
this collimated \textit{jet} of hadrons, it is possible to infer information about the 
original quark or gluon. This is very useful for probing the perturbative hard scatter.

A \textit{jet algorithm} defines how the large number of final state particle four-momenta 
are grouped into a small number of jet four-momenta. Such an algorithm should satisfy a 
number of criteria, the most important being infrared and collinear safety 
\cite{Salam:2010}. This requires that the jets are insensitive to additional soft or 
collinear emissions.

Multiple jet algorithms are implemented in the \fastjet software library \cite{FastJet}. 
In particular, \textit{sequential recombination algorithms} are popular at the \acs{LHC}, 
which iteratively combine the closest pair of particles according to some distance measure 
$d_{ij}$. 

Consider an algorithm where all the inter-particle distances $d_{ij}$ and particle-beam 
distances $d_{iB}$ are calculated. If the minimum is a $d_{ij}$, then particles $i$ and $j$
are combined into single new particle. If the minimum is a $d_{iB}$, then particle $i$ is 
declared a jet and removed from the list of particles. Then the algorithm restarts. We 
define the distances
\begin{equation}
	d_{ij} &= \min\parenths{p^{2m}_{Ti}, p^{2m}_{Tj}} \frac{\Delta R^2_{ij}}{R^2} \,,
	&& \Delta R^2_{ij} = \parenths{y_i - y_j}^2 + \parenths{\phi_i - \phi_j}^2 \\
	d_{iB} &= p^{2m}_{Ti}
\end{equation}
where $p_{Ti}$, $y_i$ and $\phi_i$ are the transverse momentum, rapidity and azimuthal 
angle of particle $i$ with respect to the beam axis, respectively. $R$ and $m$ are 
parameters of the algorithm, with $R$ effectively determining the size of the jet.

With $m=1$, known as the $k_t$ algorithm, combinations between soft particles are favoured.
This follows the evolution of \ac{QCD}, but leads to rather irregular jet shapes. 

With $m=-1$, known as the \antikt algorithm \cite{antikt}, combinations between hard 
particles are favoured. This means the jets grow outwards from a hard `seed', ultimately 
producing more circular jets. However, the jet substructure can no longer be used to infer 
details of the jet evolution history. The jets used in this thesis were \antikt jets with 
$R=0.4$, though shall be described in detail in \Section~\ref{sec:objects:jets}.
