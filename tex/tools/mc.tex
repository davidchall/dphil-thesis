%!TEX root = ../../thesis.tex

\subsection{The anatomy of an event}

\ac{MC} event generators provide fully-exclusive hadron-level simulation of \pp collision 
events at the \acs{LHC} \cite{MCnet:general}. \Figure~\ref{fig:mcevent} shows how event 
generation is factorised into several components, each describing a certain regime of 
momentum transfer.

\begin{figure}[b]
	\includegraphics[width=\mediumfigwidth]{tex/tools/event}
	\caption{Schematic diagram of a simulated \ttH event, showing how factorisation allows 
	the physics at different scales of momentum transfer $Q$ to be treated independently 
	\cite{MCnet:MatchingLectures}.
	At high-$Q$ is the hard scatter (red circle). As the scale evolves down, partons are 
	radiated in the initial state (blue) and final state (red). At low-$Q$, incoming 
	partons are confined to the beam protons, while outgoing partons hadronise (light 
	green blobs). The underlying event contains multiple partonic interactions (purple 
	blob) and beam remnants (light blue blobs). Photons (yellow) are also radiated.}
	\label{fig:mcevent}
\end{figure}

\begin{description}
\item[Hard scatter] \hfill \\
	The high scale process is selected by the user (\eg Higgs boson production via 
	gluon-gluon fusion). The relevant parton-level \acp{ME} are calculated using fixed 
	order perturbative QCD, either by the event generator itself or an external program. 
	These \acp{ME} are usually \ac{LO}, though possible improvements are discussed in 
	\Section~\ref{sec:mc:merging} and \Section~\ref{sec:mc:matching}.
\item[Parton Distribution Functions (PDFs)] \hfill \\
	Incoming parton momenta are sampled from a proton \ac{PDF}, usually probed at the 
	scale of the hard scatter $\mu_F = Q$. The LHAPDF interface \cite{LHAPDF} provides 
	access to the \acp{PDF} of several fitting collaborations, such as CTEQ \cite{CTEQ} 
	and MSTW \cite{MSTW}.
\item[\ac{FSR}] \hfill \\
	Soft and collinear radiation from outgoing partons is simulated by a universal parton 
	shower, evolving the scale from the hard scatter to the hadronisation scale of 
	\about\unit{1}{\GeV}. The successive emissions are ordered to avoid double-counting --
	common order parameters are virtuality, transverse momentum and opening angle.

	For the correct treatment of soft emissions, it is vital to preserve coherence. This 
	is inherent in an angular ordered shower, but must be manually implemented otherwise. 
	Alternatively, a \textit{dipole shower} considers emissions from colour-connected 
	pairs of partons, and is also inherently coherent.
\item[\ac{ISR}] \hfill \\
	Soft and collinear radiation from incoming partons is similarly described by a parton 
	shower. However, the small probability of evolving two partons with the kinematics 
	required by the hard process necessitates a \textit{backwards evolution}. Thus, the 
	probability that a parton originated from one of higher momentum and lower scale is 
	calculated, rather than an emission probability.
\item[Hadronisation] \hfill \\
	The confinement of partons to hadrons is non-perturbative, and must be described by a 
	hadronisation model. The \textit{string model} stretches strings between colour 
	partners. At some distance it becomes favourable to convert the potential energy to a 
	\HepProcess{\Pquark \APquark} pair, breaking the string. Once there is insufficient 
	energy to create \HepProcess{\Pquark \APquark} pairs, the hadrons `freeze out'. The 
	\textit{cluster model} splits gluons into \HepProcess{\Pquark \APquark} pairs, which 
	group into colourless clusters with a mass spectrum predicted by \ac{QCD}. These 
	clusters then decay to the physical hadrons. These hadronisation models require 
	tuning to experimental data.
\item[Hadron and \Ptau decays] \hfill \\
	Many of the hadrons produced during hadronisation are unstable, and must be decayed to
	particles that are stable on a detector timescale, while observing conservation laws. 
	Similarly, the \Ptau lepton must be decayed, hadronically or leptonically.
\item[\ac{MPI}] \hfill \\
	The \textit{underlying event} (UE) comprises additional soft hadronic activity caused 
	by partons inactive in the hard scatter. This exists because the elastic 
	parton-parton scattering cross section is sufficiently large to produce 
	\textit{multiple partonic interactions} (MPI). This activity is correlated to the 
	scale of the hard scatter. 

	In order to calculate the number of additional interactions, the spatial distribution 
	of partons within the proton must be modelled, the impact parameter of the \pp 
	collision must be known, and an IR cut-off must be imposed. This requires 
	non-perturbative models that must be tuned to experimental data.
\item[\acs{QED} radiation] \hfill \\
	Electrically charged particles can emit photons throughout the event generation.
\end{description}



\subsection{Summary of event generators}
\label{sec:mc:generators}

Three event generators are commonly used at the \acs{LHC}, mainly differing in their 
choice of hadronisation and \ac{MPI} models and parton shower order parameter. Efforts to 
rewrite the older Fortran-based programs in \cpp has led to a generation of `out-of-date' 
programs that are no longer actively developed. Even so, they are still in common usage, 
and so are included in the descriptions below.

\begin{description}
\item[Herwig] \hfill \\
	\fherwig (Fortran) \cite{fHerwig} and \herwigpp (\cpp) \cite{Herwig++} both employ an 
	angular ordered parton shower and a cluster hadronisation model. An \ac{MPI} model is 
	included in \herwigpp, but in \fherwig this was provided by the separate program 
	\jimmy.
\item[Pythia] \hfill \\
	\pythia{6} (Fortran) \cite{Pythia6} and \pythia{8} (\cpp) \cite{Pythia8} both use a 
	string hadronisation model and an advanced \ac{MPI} model. \pythia{8} uses a dipole 
	shower ordered in transverse momentum, whereas \pythia{6} offered a choice of 
	virtuality and transverse momentum ordered parton showers with coherence implemented 
	manually.
\item[Sherpa] \hfill \\
	\sherpa (\cpp) \cite{Sherpa} uses a dipole shower ordered in transverse momentum, 
	which is convenient for multi-leg merging (see \Section~\ref{sec:mc:merging}). It 
	uses a cluster hadronisation model and \iac{MPI} model similar to that of \pythia{8}.
\end{description}



\subsection{Multi-leg merging}
\label{sec:mc:merging}

Although a parton shower excellently describes the emissions of large numbers of soft and 
collinear partons, it fails to accurately model hard and isolated emissions. It can be 
desirable to describe these using fixed order \acp{ME}, which are better suited to the 
task.

There are a couple of immediate issues that need resolving however. First, we require a 
smooth transition from emissions of \iac{ME} to those of the parton shower. Second, each 
\ac{ME} is inclusive, and attempting to combine \acp{ME} of differing multiplicity 
naturally leads to problems of double counting.

By using a merging prescription, such as the CKKW-L algorithm \cite{CKKW,Lonnblad:2002} 
employed by \sherpa, it is possible to consistently combine \ac{LO} matrix elements 
with differing multiplicities, whilst matching to the parton shower correctly. This does 
require the introduction of a merging scale though. Loosely speaking, this scale separates
the \ac{ME} and parton shower descriptions of the emissions.



\subsection{NLO matching}
\label{sec:mc:matching}

It is also possible to match \iac{NLO} \ac{ME} to a parton shower, to improve the accuracy
of both the normalisation and shape of observables \cite{Nason:2012}. Such a calculation 
must include the \ac{LO}, virtual-loop and real-emission diagrams, while mapping smoothly 
onto the parton shower for soft emissions. There are two valid prescriptions:

\begin{description}
\item[\mcatnlo] \hfill \\
	Simply adding a parton shower to \iac{NLO} \ac{ME} introduces double counting. The 
	\mcatnlo method compensates for this overlap in the \ac{NLO} calculation. This 
	renders the \ac{ME} dependent upon the parton shower used in the \ac{MC} event 
	generator. It also introduces the possibility of negatively weighted events.

	Originally implemented in the program \mcatnlo for matching to \fherwig 
	\cite{MCatNLO-Herwig} and \herwigpp \cite{MCatNLO-Herwig++}, the method has now been 
	automated within the \amcatnlo program and extended for use with \pythia{6} and 
	\pythia{8} \cite{aMCatNLO,MCatNLO-Pythia}. A variant is also within \sherpa.
\item[POWHEG] \hfill \\
	The \powheg method requires that the hardest emission is always generated by the 
	\ac{ME}. It achieves the correct hard and soft behaviour by convolving the \ac{LO} 
	\ac{ME} with a modified Sudakov factor, and then reweights the differential cross 
	section to the \ac{NLO} result. Thus the \ac{ME} is independent of the subsequent 
	parton shower. However, if the parton shower is not transverse momentum ordered, it is
	necessary to use truncated and vetoed parton showers to correctly fill the phase 
	space.

	Originally implemented in \powhegbox \cite{Powheg-method,Powheg-method2,PowhegBox}, 
	variants are now included in \herwigpp and \sherpa.
\end{description}



\subsection{Additional considerations}
\label{sec:mc:other}

\begin{description}
\item[Detector simulation] \hfill \\
	In order to compare \ac{MC} events to experimental events recorded at the \acs{LHC}, 
	it is vital to simulate how the outgoing particles interact with the detector. This 
	is also necessary to calibrate the detector response and estimate efficiencies. 
	\geant \cite{GEANT4,ATLAS-simulation} is used to simulate the energy deposition of 
	each particle during its trajectory through the ATLAS detector (see 
	\Chapter~\ref{chap:experiment}). Since long-lived particles will decay \textit{en 
	route}, particles with lifetime \unit{$c\tau > 10$}{\milli\metre} are decayed by 
	\geant rather than the \ac{MC} generator.

	\textit{Digitisation} converts the energy deposition into readout voltages and 
	currents. From this point, the events can be treated like experimental collision 
	events.
\item[Pile-up simulation] \hfill \\
	As described in \Chapter~\ref{chap:experiment}, each \acs{LHC} bunch crossing can 
	result in soft proton-proton interactions in addition to the hard process, known as 
	\textit{pile-up}. This obscures the interesting physics and is important to model 
	accurately.

	In-time pile-up (same bunch crossing as hard process) is modelled by overlaying 
	simulated energy deposits from soft \pp interactions generated with \pythia{8}. The 
	number of overlaid events depends upon the beam conditions (see 
	\Section~\ref{sec:dataset}).

	Out-of-time pile-up (different bunch crossing from hard process) affects detector 
	sub-systems whose latency is longer than the bunch spacing. These sub-systems have 
	further pile-up events overlaid; again this depends on the beam conditions.
\end{description}
