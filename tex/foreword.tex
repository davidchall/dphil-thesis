%!TEX root = ../thesis.tex

The Large Hadron Collider was built with the primary goal of determining the mechanism for 
electroweak symmetry breaking. The leading theory, the Brout-Englert-Higgs mechanism, 
predicted the presence of a new scalar particle known as the Higgs boson, with couplings to 
other particles determined by their masses. The mass of the Higgs boson itself was unknown, 
but was indirectly constrained to a narrow range by other precision measurements. By the time 
the first beam circulated in 2008, large teams of researchers on the ATLAS and CMS 
experiments were preparing for the search, with data from the Tevatron collider in the United 
States steadily reducing the allowed Higgs boson mass range.

Less than two weeks after the first LHC beam, a faulty connection short-circuited a section 
of superconducting magnets, ripping them from their cement base. This incident delayed first 
collisions by more than a year, and to protect the LHC the centre of mass energy was cut by 
half. The Higgs boson production rates were thus reduced and it seemed a Higgs boson 
discovery would take years to achieve. However, the collider performed exceptionally well in 
2011 and 2012, producing data faster than anticipated. On July 4, 2012, the ATLAS and CMS 
experiments announced the discovery of a new boson, with considerably less data than 
pre-collision predictions.

The ATLAS publication describing the discovery included three Higgs boson decay channels: 
\HepProcess{\PHiggs \HepTo \Pphoton\Pphoton}, \HepProcess{\PHiggs \HepTo \PZ\PZ}, and \HWW. 
The strongest evidence for a new boson came from the \HepProcess{\PHiggs \HepTo 
\Pphoton\Pphoton} and \HepProcess{\PHiggs \HepTo \PZ\PZ} channels, for which the observed 
rates were somewhat higher than expected, whereas the rate in the \HWW decay channel was 
somewhat lower than expected. After the discovery, the LHC more than doubled the data set, 
and the rates in the three channels became more compatible with expectation. In early 2013 
ATLAS published these rate measurements, as well as measurements demonstrating that the 
discovered boson was most likely a spin-0 particle -- a crucial requirement for a Higgs boson.

In order to improve the precision of these measurements and that of the Higgs boson mass, 
ATLAS dedicated the following year to upgrading its analyses. The \HepProcess{\PHiggs \HepTo 
\Pphoton\Pphoton} and \HepProcess{\PHiggs \HepTo \PZ\PZ} channels provided the most precise 
measurement of the Higgs boson mass, while the \HWW channel gave the most precise 
determination of the production rate. In each channel the signal significance was above the 
threshold for discovery. In addition, evidence for the predicted \HepProcess{\PHiggs \HepTo 
\Ptau\Ptau} decay was achieved.

David Hall's thesis presents the final ATLAS measurement of the \HWW decay in gluon-fusion 
production using the 2011 and 2012 data sets, taking the reader through the process of 
discovery and measurement. A precursor to the observation of the \HWW decay was the 
measurement of the dominant background of non-resonant \WW production. This measurement 
shared many of the features of the \HWW measurement; one important feature was the removal or 
separation of events with one or more jets. A zero-jet requirement effectively eliminated the 
large top-quark background, but it introduced an uncertainty on the efficiency of this jet 
veto. David was responsible for a data-based correction to reduce this uncertainty, and for 
evaluating the residual uncertainty. The details of the procedure, and of the full 2011 \WW 
cross section measurement, are provided in his thesis.

In the \HWW measurement events are separated by jet multiplicity, as events with one jet 
contribute significantly to the sensitivity. In the analysis one must determine the 
theoretical uncertainty on the migration of events between jet bins. A covariance matrix can 
be constructed to describe the normalisation and migration uncertainties, and different 
procedures for their evaluation make different assumptions about this matrix. David studied 
these issues in detail and evaluated the uncertainties associated with a method that had not 
been previously used by the experiments, but which accommodated the highest precision 
calculations and thus led to a reduction in the uncertainty associated with jet binning. His 
thesis details this method and provides a comparison of covariance matrices for different 
uncertainty estimation procedures.

A final notable aspect of the thesis is the clear and detailed description of the backgrounds 
to the \HWW search and measurement. The dominant non-resonant \WW background is normalised to 
data using a control region; Monte Carlo simulation is then used to model the signal region, 
with corresponding theoretical uncertainties. David worked directly on evaluating these 
uncertainties and describes them in detail. A smaller background is a \PW boson produced in 
association with an off-shell photon that splits into a pair of leptons. When only one of 
these leptons is reconstructed, it can mimic the leptonic decay of a \PW boson and serve as a 
background to the signal process. David performed extensive studies to ensure this process 
was accurately simulated, and details are given in his thesis.

Measuring the Higgs boson in the \WW decay channel presents many challenges, which are 
apparent as one reads David's thesis. The solutions to these challenges represent an 
impressive body of work from many people over many years. Collecting them into a thesis with 
many unique insights, David provides a clear exposition of the final word on the ATLAS \HWW 
measurement in the discovery data set.

\vspace{12pt}\noindent
Oxford, March 2015 \hfill Dr Chris Hays
