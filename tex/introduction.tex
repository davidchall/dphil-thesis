%!TEX root = ../thesis.tex

This thesis describes the search, discovery and measurement of the Higgs boson using 
proton-proton collision data recorded by the ATLAS experiment at CERN. This is accomplished 
by searching for collisions where a Higgs boson is produced and subsequently decays to two 
\PW bosons, each of which decay to an electron or muon and a neutrino (\ie \HWWlvlv). This 
search suffers from large experimental backgrounds, such as continuum \WW production, which 
must be accurately modelled to yield sensitivity to the Higgs boson.

First, the theoretical motivation for the Higgs boson is presented in 
\Chapter~\ref{chap:motivation}. Then, \Chapter~\ref{chap:tools} outlines some important 
concepts related to making precise predictions within the Standard Model, which shall be 
referred to throughout the thesis. The experimental setup of the LHC and the ATLAS detector 
are described in \Chapter~\ref{chap:experiment}.

Focus then moves to the data analysis itself. \Chapter~\ref{chap:selection} offers an 
overview of the entire \HWW analysis, detailing the selection of Higgs boson signal events 
and the rejection of backgrounds. Following this, signal modelling is described in 
\Chapter~\ref{chap:signal}, \WW background modelling is described in \Chapter~\ref{chap:ww} 
(including a dedicated cross section measurement), and the modelling of other backgrounds is 
described in \Chapter~\ref{chap:backgrounds}. The experimental results are presented and 
discussed in \Chapter~\ref{chap:results}. Finally, in Chapters~\ref{chap:conclusions} and 
\ref{chap:thesis_conclusions}, we draw conclusions from the results of this analysis and of 
others conducted simultaneously at the LHC, and consider the outlook of Higgs physics.
