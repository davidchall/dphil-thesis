%!TEX root = ../thesis.tex

The Standard Model of particle physics describes the behaviour of sub-atomic particles. 
Since its formulation in the 1970s, it has experienced unparalleled success in modelling a 
wide range of phenomena; no experimental result within the remit of the Standard Model is 
currently considered to significantly contradict its validity.\footnote{
	Observation of neutrino oscillations required neutrino masses to be manually added to 
	the Standard Model. It is widely believed that their relatively small masses will be 
	explained by new physics.
}
However, there are a number of physical phenomena that the Standard Model is unable to 
describe: gravitational attraction between massive objects, the observed asymmetry between 
matter and antimatter in the Universe, and astronomical evidence for dark matter and the 
cosmological constant.

A crucial aspect of the Standard Model is how non-zero masses are imparted to fundamental 
particles. These are forbidden by underlying symmetries of the theory, though remain an 
experimental fact; for example, atoms could not form if the electron did not possess mass.
This is achieved via interactions with a ubiquitous Higgs field, excitations of which 
correspond to Higgs bosons. As the only undiscovered particle of the Standard Model, the 
discovery of the Higgs boson is of utmost importance to particle physics: it would complete 
our knowledge of the Standard Model, and in particular confirm the mechanism of mass 
generation. As such, it was a primary goal of the LHC physics program, which began in 2010.

This thesis describes the search, discovery and measurement of the Higgs boson using 
proton-proton collision data recorded by the ATLAS experiment at CERN. This is accomplished 
by searching for collisions where a Higgs boson is produced and subsequently decays to two 
\PW bosons, each of which decay to an electron or muon and a neutrino (\ie \HWWlvlv). This 
search suffers from large experimental backgrounds, such as continuum \WW production, which 
must be accurately modelled to yield sensitivity to the Higgs boson.

First, the theoretical motivation for the Higgs boson is presented in 
\Chapter~\ref{chap:motivation}. Then, \Chapter~\ref{chap:tools} outlines some important 
concepts related to making precise predictions within the Standard Model, which shall be 
referred to throughout the thesis. The experimental setup of the LHC and the ATLAS detector 
are described in \Chapter~\ref{chap:experiment}.

Focus then moves to the data analysis itself. \Chapter~\ref{chap:selection} offers an 
overview of the entire \HWW analysis, detailing the selection of Higgs boson signal events 
and the rejection of backgrounds. Following this, signal modelling is described in 
\Chapter~\ref{chap:signal}, \WW background modelling is described in \Chapter~\ref{chap:ww} 
(including a dedicated cross section measurement), and the modelling of other backgrounds is 
described in \Chapter~\ref{chap:backgrounds}. The experimental results are presented and 
discussed in \Chapter~\ref{chap:results}. Finally, in \Chapter~\ref{chap:conclusions}, we 
draw conclusions from the results of this analysis and of others conducted simultaneously at 
the LHC, and consider the outlook of Higgs physics.
